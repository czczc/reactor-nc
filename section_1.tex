%!TEX root = reactor_nc_main.tex
%%%%%%%%% Section: Intro %%%%%%%%%
\section{Introduction: Neutrino oscillations and nuclear reactors} 
\label{sec:intro}

The Standard Model of electroweak interactions,
developed in late 1960s, incorporates neutrinos as left-handed partners of the charged leptons. Since the weak interactions are the
only way neutrinos interact with anything, the unneeded right-handed components of the neutrino field are absent 
by definition and neutrinos are assumed to be massless with the individual lepton numbers strictly conserved.
This assignment was supported by the lack of observation of decays like 
$\mu^+ \rightarrow e^+ + \gamma$ or $K_L \rightarrow e^{\pm} + \mu^{\mp}$,
despite the long tradition of efforts to search for them.

The discovery over the past several decades
of neutrino oscillations proved that these assumptions were incorrect; it
represents one of the very few instances that show that the otherwise
extremely successful Standard Model is incomplete. It means that
neutrinos have a finite mass, albeit very small, and that the lepton flavor is not a conserved quantity.
Box 1 explains the basic physics of neutrino oscillations and their relation with neutrino masses. It
also introduces the parameters used in the oscillation formalism. Determination of all their values,
with ever increasing accuracy,
was and continues to be the main goal of the experiments.
The current experimental values of the mass-squared differences $\Delta m^2_{ij}$ and of the mixing angles $\theta_{ij}$ can be found
in the latest editions of the Review of Particle Physics~\cite{PDG14}.
Historically, the concept of neutrino oscillations was first considered by Pontecorvo~\cite{Pontecorvo57, Pontecorvo58}
and by Maki, Nakagawa and Sakata~\cite{MNS62}, hence the neutrino mixing matrix is usually called the PMNS matrix.

Study of reactor neutrinos played a very significant part in the discovery and detailed study of neutrino oscillations and will continue 
to be essential to its further progress. Here we briefly review
the main points of this saga. Fig.~\ref{fig:intro1} illustrates how the flavor composition of the reactor neutrino flux, for neutrinos of 4 MeV energy
used as an example, is expected to oscillate as a function of the distance. Experimental verification of this behavior, and the quantitative analysis of the
results are the main topics discussed below.

\begin{figure}[htb]
\begin{centering}
\includegraphics[width=\columnwidth]{figs/osci.pdf}
\par\end{centering}
\caption{\label{fig:intro1} Flavor composition of the 4 MeV energy reactor neutrino flux as a function of the distance  $L$. }
\end{figure}

Existence of neutrinos was predicted by Pauli already in 1930~\cite{Pauli30} in his famous letter attempting to explain the continuous electron
energy distribution in the nuclear beta decay. Only in 1953--1959 Reines and Cowan~\cite{Reines53,Cowan56,Reines59} were able to show that neutrinos
were real particles. Their observation used the electron antineutrinos emitted by a nuclear reactor and started a long tradition of fundamental
discoveries using the reactor $\bar{\nu}_e$'s.

In the early experiments detectors were placed at distances $L \le 100$ m~\cite{ILL,Gosgen,Rovno,Krasnoyarsk,SRP,Bugey4,Bugey3} (for a review see~\cite{Bemporad02}). As expected, no variations
with the distance were observed, but these pioneering experiments were important for the understanding of the reactor spectrum, discussed
in the next section. The KamLAND experiment~\cite{Kamland03,Kamland05,Kamland08} in 2000s, discussed in more detail in Section III, has convincingly shown that the earlier
solar neutrino measurements are indeed  caused by oscillations. It demonstrated that the reactor neutrinos indeed
oscillate, i.e. that the $\bar{\nu}_e$ component changes as predicted with $L/E_{\nu}$. It also allowed the most accurate determination of the 
mass-squared difference $\Delta m^2_{21}$.  

In the next generation of reactor experiments, Daya Bay~\cite{Dayabay,Dayabay14}, RENO~\cite{Reno}  and Double Chooz~\cite{DChooz,DChooz14}, the longstanding puzzle of the
value of the mixing angle $\theta_{13}$ was successfully resolved; it turns out that its value $\theta_{13} \sim 8.4^\circ$
is not as small as many physicists expected. That discovery, described in Section IV, opened opportunities for further experiments that should  
eventually allow us to determine the so-far missing fundamental features of the oscillations, the neutrino mass hierarchy and the phase $\delta_{CP}$ that characterizes the possible CP (charge and parity) violation. 
The planned medium-baseline reactor experiments, described in Section V,  promises to be an important step on that path.

Finally, most of the oscillation results are well described by the simple three-neutrino generations hypothesis. However, there are few anomalous indications,
the so-called reactor antineutrino anomaly~\cite{Mention} among them, that cannot be explained this way. If confirmed, they would indicate the existence of additional
fourth or more neutrino families called sterile neutrinos. They lack  weak interactions and are observable only due to their mixing with the familiar active neutrinos.  Very short baseline
reactor experiments, discussed in Section VI, might decide whether this fascinating possibility is realistic or not.

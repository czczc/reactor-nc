%!TEX root = reactor_nc_main.tex
%%%%%%%%% Section: Mass Hierarchy: JUNO %%%%%%%%%
\newcommand{\fixit}[1]{{\color{red}FIXIT: #1}}
\section{Determination of Neutrino Mass Hierarchy: JUNO}

At present, only the absolute value of the neutrino mass differences $\Delta m^2_{32}$ and $\Delta m^2_{31}$ are known, not their sign. The sign of $\Delta m^2_{31}$ has impacts to many important fundamental topics in particle physics, astrophysics and cosmology. Depending on $\Delta m^2_{31}>0$ or $\Delta m^2_{31}<0$, the neutrino mass ordering is usually referred as normal or inverted mass hierarchy (MH), respectively. The reactor $\bar\nu_e$-oscillation frequency is modulated by $\Delta m^2_{31}$ and $\Delta m^2_{32}$. At a medium baseline of $\sim$60 km, multiple $\theta_{13}$ oscillation peaks are clearly visible on top of the $\theta_{12}$ oscillation, as shown in Figure~\ref{fig:juno}. The MH information can be extracted from the oscillation pattern, by using Fourier transform method~\cite{Zhan-PRD08,Zhan-PRD09} or normal $\chi^2$ analysis~\cite{Li-PRD13}. In addition, the effective mass-squared differences measured by medium-baseline reactor experiment and long-baseline muon neutrino disappearance experiment are different combinations of $\Delta m^2_{31}$, $\Delta m^2_{31}$ and other oscillation parameters. This would provide new information regarding the neutrino MH.

\begin{figure}[!htb] \label{fig:juno}
  \centering
  \includegraphics[width=\columnwidth]{figs/juno.pdf}
  \caption{The reactor $\bar\nu_e$ spectra at a baseline of 60 km in $L/E$ space for no oscillation and $P(\bar\nu_e\to\bar\nu_e)$ oscillation in the cases of normal and inverted neutrino mass hierarchy, assuming $\sin^22\theta_{13}=0.084$.}
\end{figure}

Jiangmen Underground Neutrino Observatory (JUNO), known as Daya
Bay II before, is a multi-purpose experiment with the primary goals to resolve the mass hierarchy issue and precision measurements of neutrino oscillation parameters. JUNO also provide great physics opportunities
in other topics, e.g, observe neutrinos from supernova, Earth interior and Sun, search atmospheric and sterile neutrinos and perform other exotic searches. JUNO locates in Kaiping city, Guangdong province, in south of China, about 150 km west of Hong Kong. The JUNO detector is underground at 480 m below sea level, plus a mountain above it, the total vertical overburden is 1800 m.w.e. JUNO observes $\bar\nu_e$ from Yangjiang nuclear power plant (NPP) and Taishan NPP at a baseline of $\sim$53 km, near the maximum $\theta_{12}$ oscillation. The Yangjiang NPP has six reactors cores of 2.9 GW$_{th}$ thermal power and the Taishan NPP has planned four cores of 4.59 GW$_{th}$, both are under construction. Unequal baselines could cancel the oscillation structure, thus baseline difference is controlled within 500 m to prevent significant degradation of the MH discrimination power~\cite{Li-PRD13}.

The JUNO detectors are under design. The preliminary design is that a central detector submerged in a water pool with muon trackers installed on the top of water pool. The water pool is equipped with PMTs and acts as an active muon veto, and it also provides passive shielding from the natural radioactivities from the surrounding rock and air. The top muon trackers can provide complementary track measurement to the cosmic muons that miss the central detector. The central detector of JUNO consists of 20 kton of un-doped liquid scintillator (LS), contained in a spherical acrylic tank ("Acrylic Sphere" option), or a nylon or other polymer film balloon ("Balloon" option). For the "Acrylic Sphere" option, the acrylic tank is supported by a double layer stainless steel strut. The buffer liquid between the acrylic tank and strut is water, which is physically connected with but optically separated from the outside water Cherenkov detector. For the "Balloon" option, the buffer liquid is non-scintillation LAB contained in a stainless steel sphere of a radius $\sim $40 m. The PMTs are installed on the inner surface of the stainless steel vessel. For both options, the PMTs will be protected from implosion. Taking into account the implosion container and mechanical clearance, the photocathode coverage can reach 75\%-78\% for various options.

Energy resolution is key to JUNO. Th mass hierarchy discrimination power decreases as the energy resolution gets worse~\cite{Li-PRD13}.
It has been found that a $3\%/\sqrt{E}$ energy resolution is essential. Such energy resolution is demonstrated to be achievable by the Monte Carlo simulation, assuming a maximum possible photocathode coverage and a couple of technical improvements: 1) use high quantum efficiency (QE) PMTs with the maximum QE to be 35\%; 2) improve the LS attenuation length to 20 m at 430nm, which consists of an absorption length of 60 m and a Rayleigh scattering length of 30 m. A new type of 20 inch, high efficiency PMT is being developed by JUNO. Instead of using the traditional dynode, the new PMT design uses micro-channel plate (MCP) to have nearly 4$\pi$ collection of the photoelectrons. Using the super bialkali technic, the quantum efficiency of the photocathode is expected to reach to 35\% high. To increase the optical transparency of LS, purification of the raw solvent and fluors at the production stage and online purification such as Al$_2$O$_3$ column filtration at the operation stage is effective. A position resolution of \fixit{?cm}/$\sqrt{E (MeV)}$ is achievable by using time-based likelihood reconstruction method.

%Other effects like the uncertainties from non-uniformity correction, PMT dark rates, electronics noise, etc, would also affect the energy resolution and should be carefully considered during the detector design.

Unlike the $\theta_{13}$ reactor experiments, JUNO uses un-loaded scintillator to reduce the risk in the production, long-term operation, as well as the online purification of a massive detector. Increasing the light yield and optical transparency of the liquid scintillator is important for improving the energy resolution. Maximum light yield can be achieved by optimizing the concentration of fluors. The other issue is to reduce the radioactive impurities in LS, particularly for the potential research of solar neutrinos. An online purification system with distillation is essential to remove U/Th and lead-210 from LS.

%In JUNO, the mass hierarchy information is extracted from the oscillation structure in the energy spectrum.
Three main effects can cause non-linear energy response in a LS detector: quenching effect, Cerenkov process and possible electronics non-linear response. If the energy non-linearity correction has large uncertainties, particular residual non-linearity shape can fake the oscillation pattern with a wrong mass hierarchy~\cite{Qian-PRD13}. The energy scale uncertainty need be controlled within 1\%, which is expected to be achievable in JUNO since it has been achieved in a smaller detector like Daya Bay~\cite{Zhang-Neutrino14}. It is found that the multiple $\theta_{13}$ oscillation peaks in the measured $L/E$ spectrum can be utilized to constrain the non-linearity response formula~\cite{Li-PRD13}.

With full operation of the reactors from Yangjiang and Taishan NPP, the expected signal at JUNO is 80 reactor $\bar\nu_e$ events per day. The delayed signal is a single 2.2-MeV $\gamma$ released by neutron capture on protons with a capture time of $\sim$200 microseconds. Though the natural radioactivity from detector materials can contaminate the delayed signal, coincidences in energy, position and time can suppress the accidental background to $\sim$2\% of the $\bar\nu_e$ candidates. The cosmogenic $\beta$-n emitters $^9$Li and $^8$He is the other dominant background at JUNO, due to large detector size and relative high cosmic muon rate. Similar as previous generation reactor neutrino experiments, the rest of the backgrounds at JUNO include fast neutrons, the $^{13}$C$(\alpha, n)^{16}$O background, geo-neutrinos.

The sensitivity of mass hierarchy determination at JUNO can exceed 3$\sigma$ after taking into account the realistic locations of the reactor cores, uncertainties of the energy non-linearity, etc. Assume the effective mass-squared difference measured by accelerator experiments can reach to $\sim$1.5\% or 1\% precision~\cite{Agarwalla}, then it can help to improve sensitivity to be 3.7$\sigma$ or 4.4$\sigma$\cite{Li-PRD13}. Besides, precision measurement of the neutrino oscillation parameters allows testing unitarity of the neutrino mixing matrix, which is of great importance. For $\Delta m^2_{21}$, $\Delta m^2_{32}$ and $\sin^2\theta_{12}$, JUNO expects to measure them with a precision better than 1\%. Considering the precision of $\sin^2\theta_{13}$ can be measured to $\sim4\%$ by the on-going $\theta_{13}$ reactor experiments, the unitarity of the neutrino mixing matrix can be probed to 1\% level.

Measuring mass hierarchy at JUNO with reactors is complementary in physics since it's independent on $\theta_{23}$ and the unknown CP phase. The technical challenges are significant to build a 20kton liquid scintillator detector with high precision and excellent energy resolution. JUNO expects to start data taking in 2020.

%!TEX root = reactor_nc_main.tex
%%%%%%%%% Section: Mass Hierarchy: JUNO %%%%%%%%%
\newcommand{\fixit}[1]{{\color{red}FIXIT: #1}}
\section{Determination of Neutrino Mass Hierarchy: Medium Baseline Reactor Experiments}

At present, only the absolute values of the neutrino mass-squared differences $\Delta m^2_{32}$ and $\Delta m^2_{31}$ are known, not their sign. 
Depending whether $\Delta m^2_{31}>0$ or $\Delta m^2_{31}<0$, the neutrino mass ordering is usually referred as normal or inverted mass hierarchy, respectively. 
% The neutrino mass hierarchy has important impacts on many important fundamental topics in particle physics, astrophysics and cosmology. 
The neutrino mass hierarchy (MH) is a problem of the fundamental importance~\cite{MHwhitepaper}.
Determination of the MH represents an important step in the formulation of the Generalized Standard Model of particle physics. 
It will reduce the uncertainty in the experiments aiming at the measurement of the CP-violating phase. 
It will help in defining the goals of the forthcoming neutrinoless double beta decay experiments.
It will also improve our understanding of the core-collapse supernovae.

The reactor $\bar\nu_e$-oscillation frequency is modulated by $\Delta m^2_{31}$ and $\Delta m^2_{32}$. 
At a medium baseline of $\sim$60 km, multiple small-amplitude $\theta_{13}$ oscillation peaks show up on top of the long wavelength $\theta_{12}$ oscillation, as shown in Fig.~\ref{fig:intro1}.
Depending whether the MH is normal or inverted, this oscillation pattern shifts slightly.
The MH information can be extracted by using the common likelihood analysis~\cite{Li-PRD13} or the Fourier transform method~\cite{Zhan-PRD08,Zhan-PRD09}. 
Additional information regarding the neutrino MH could be obtained by combing with the long-baseline muon neutrino disappearance experiments~\cite{Minakata}, as the effective mass-squared differences measured there is a different combination of $\Delta m^2_{31}$, $\Delta m^2_{32}$ and other oscillation parameters.

Jiangmen Underground Neutrino Observatory (JUNO), currently under construction, will be a multi-purpose experiment with the primary goals to determine the neutrino mass hierarchy and precisely measure the neutrino oscillation parameters with reactors.
%JUNO also provide great physics opportunities in other topics, e.g, observe neutrinos from supernova, Earth interior and Sun, search atmospheric and sterile neutrinos and perform other exotic searches.
JUNO is located in Kaiping city, Guangdong province, in south of China, about 150 km west of Hong Kong. The JUNO detectors will be underground with a total vertical overburden of 1800 m.w.e. JUNO observes $\bar\nu_e$ from the Yangjiang nuclear power plant (NPP) and the Taishan NPP at an equal baseline of $\sim$53 km, as illustrated in Fig.~\ref{fig:juno}, near the maximal $\theta_{12}$-oscillation baseline. The Yangjiang NPP has six reactors cores of 2.9 GW$_{th}$ each and the Taishan NPP has planned four cores of 4.6 GW$_{th}$ each, both are under construction. The baseline difference to the two NPPs is controlled to less than 500 m to prevent significant degradation of the MH discrimination power~\cite{Li-PRD13}.

The JUNO detectors are being designed. The preliminary design has a central detector submerged in a water pool with the muon trackers installed on the top of the pool. 
The water pool is equipped with PMTs and acts as an active Cherenkov detector for muon veto. 
It also provides passive shielding against the natural radioactivities from the surrounding rock and air. 
The top muon trackers can provide complementary track measurements of the cosmic muons. 
The central detector of JUNO consists of 20 kton of liquid scintillator (LS), contained in a spherical acrylic tank (the ``Acrylic Sphere'' option), or a thin-film transparent balloon (the ``Balloon'' option). 
In the ``Acrylic Sphere'' option, the acrylic tank is supported by double-layered stainless steel frames. 
The buffer liquid between the acrylic tank and the supporting frames is water, which is physically connected with but optically separated from the outside water Cherenkov detector. 
In the ``Balloon'' option, the buffer liquid is non-scintillating linear-alkylbenzene (LAB) contained in a 40-m-radius stainless steel sphere, where the PMTs are mounted. 
For both options, the PMTs will be protected from implosion by acrylic enclosures. 
Taking into account the implosion container and mechanical clearance, the maximal photocathode coverage is 75\%--78\% for various options by installing $\sim$18000 PMTs.

A good energy resolution of less than $3\%/\sqrt{E(\textrm{MeV})}$ is essential for JUNO to maintain the MH discrimination ability~\cite{Li-PRD13}. 
To achieve that, from Monte Carlo simulation, beside keeping a maximal photocathode coverage, additional technical improvements are necessary: 
1) use high quantum efficiency (QE) PMTs to increase the light detection efficiency;
2) increase the light yield of the LS;
3) improve the optical transparency of the LS.
A new type of 20-inch, high-efficiency PMT is being developed by JUNO. Instead of using the traditional dynode, the new PMT design uses micro-channel plate (MCP) to have a near-4$\pi$ collection of the photoelectrons. 
Using the super bialkali photocathode, the QE is expected to reach to 35\%. 
Better light yield can be achieved by optimizing the concentration of fluors. 
Purification of the raw solvent and fluors, combining with on-line purification using Al$_2$O$_3$ column filtration, has been found to be effective in increasing the optical transparency of the LS. An attenuation length of $\sim$30 m (at 430 nm wavelength) is desired.

Calibration of the absolute energy scale is crucial. 
In particular, three main effects cause non-linear energy response for a LS detector: scintillator quenching, Cherenkov radiation and possible non-linear electronics response. 
If the energy non-linearity correction has large uncertainties, particular residual non-linear shapes may fake the oscillation pattern with a wrong mass hierarchy~\cite{Qian-PRD13}. 
The absolute energy scale uncertainty needs to be controlled to a few tenths of percent. 
This is expected to be achievable with the experience from Daya Bay~\cite{Zhang-Neutrino14}, however, requires a comprehensive calibration program for a large detector such as JUNO.
% Interestingly, the multiple $\theta_{13}$ oscillation peaks in the measured $L/E$ spectrum can be utilized to constrain the energy scale~\cite{Li-PRD13}.

JUNO expects to detect about 60 reactor $\bar\nu_e$ events per day. The measured spectrum after 6 years is shown in Fig.~\ref{fig:juno}.
The sources of background at JUNO are similar to those of KamLAND. 
However, it is challenging to veto the cosmogenic $^9$Li and $^8$He background at JUNO, due to the larger size, shallower depth and therefore higher muon rate ($\sim$3 Hz, about 15 times higher than KamLAND) at JUNO. 
The $^9$Li and $^8$He isotopes are mostly produced by the muons accompanied by large electromagnetic or hadronic showers~\cite{KamLAND-spall}. In KamLAND, if a shower muon is tagged, the whole detector is vetoed for 2 s. Such a veto strategy could lead to a significant signal loss at JUNO. 
Since the lateral distance of the isotopes from the parent muon trajectory is roughly exponential~\cite{KamLAND-spall}, a proper cylindrical veto along the muon track can efficiently remove the background with minimal loss of signals. Thus, the ability to track the shower muons in JUNO is essential, which demands new developments in the muon veto system as well as the simulations and reconstructions.

\begin{figure}[tb] \label{fig:juno}
  \centering
  \includegraphics[width=\columnwidth]{figs/juno.pdf}
  \caption{{\bf The layout of JUNO and the expected energy spectrum assuming 6 years' running.} JUNO is located at an equal baseline of $\sim$53 km from the powerful reactors at Yangjiang and Taishan. The shaded histograms in the insert show the simulated $\bar\nu_e$ energy spectra at JUNO with and without backgrounds.}
\end{figure}

The sensitivity of the mass hierarchy determination at JUNO is expected to exceed 3$\sigma$ (for the statistical interpretations see~\cite{Qian-Stat,Blennow}) after taking into account the realistic systematic uncertainties~\cite{Li-PRD13,MBRwitepaper}. 
Assuming that the effective mass-squared difference measured by the ongoing accelerator experiments can achieve 1.5--1\% precision~\cite{Agarwalla}, the MH sensitivity at JUNO can be improved to 3.7--4.4$\sigma$~\cite{Li-PRD13}. 
In addition, JUNO has great potentials in the precision measurements of the neutrino oscillation parameters. JUNO expects to measure $\Delta m^2_{21}, |\Delta m^2_{31}|$ and $\sin^2 \theta_{12}$ to precisions better than 1\%. This offers a major step toward the unitarity test of the neutrino mixing matrix~\cite{unitarity13} and is important to guide the directions of future experiments and theories. The JUNO collaboration currently consists of 48 institutions from 9 countries or regions, with a total number of 329 members. JUNO expects to start data taking in 2020.

RENO-50~\cite{RENO-50}, a proposed reactor experiment in Korea, has similar scientific goals as JUNO. RENO-50 plans to build an underground detector consisting of 18 kton ultra low-radioactivity liquid
scintillator and 15,000 20-inch PMTs, at $\sim$50 km away from the Hanbit (Yonggwang) nuclear power plant in Korea. RENO-50 expects to start data taking around 2019--2020.

The next-generation medium baseline reactor experiments, JUNO and RENO-50, provide a unique opportunity to determine the neutrino mass hierarchy with the precision measurement of the reactor neutrino spectrum.
Most systematic effects are well-understood and studied, although the technical challenges are significant. 
The MH sensitivity is expected to reach 3--4$\sigma$.
The reactor measurements are independent of $\theta_{23}$, the CP phase, and the matter effect. 
They will be complementary to the future long-baseline accelerator~\cite{LBNE} and atmospheric~\cite{PINGU} neutrino oscillation programs. 




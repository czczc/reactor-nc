%!TEX root = reactor_nc_main.tex
%%%%%%%%% Section: Mass Hierarchy: JUNO %%%%%%%%%
\newcommand{\fixit}[1]{{\color{red}FIXIT: #1}}
\section{Determination of Neutrino Mass Hierarchy: JUNO}

At present, only the absolute value of the neutrino mass differences $\Delta m^2_{32}$ and $\Delta m^2_{31}$ are known, not their sign. Depending whether $\Delta m^2_{31}>0$ or $\Delta m^2_{31}<0$, the neutrino mass ordering is usually referred as normal or inverted mass hierarchy (MH), respectively. The neutrino mass hierarchy has important impacts on many important fundamental topics in particle physics, astrophysics and cosmology~\cite{MHwhitepaper}. Determination of the mass hierarchy represents an important step in the formulation of the Generalized Standard Model of particle physics. It will also reduce the uncertainty in the experiments aiming at the measurement of the CP-violating phase. It will help in defining the goals of the forthcoming neutrinoless double beta decay experiments, It will also improve our understanding of the core-collapse supernovae. In short, it is a problem of the fundamental importance.

The reactor $\bar\nu_e$-oscillation frequency is modulated by $\Delta m^2_{31}$ and $\Delta m^2_{32}$. At a medium baseline of $\sim$60 km, multiple $\theta_{13}$ oscillation peaks are clearly visible on top of the long wavelength $\theta_{12}$ oscillation.
%, as shown in Figure~\ref{fig:juno}.
The MH information can be extracted from the oscillation pattern, by using the usual $\chi^2$ analysis~\cite{Li-PRD13} or Fourier transformation method~\cite{Zhan-PRD08,Zhan-PRD09}. Additional new information regarding the neutrino MH could also be obtained by combing with the long-baseline muon neutrino disappearance experiments, as the effective mass-squared differences measured there is a different combination of $\Delta m^2_{31}$, $\Delta m^2_{32}$ and other oscillation parameters.

\begin{figure}[!htb] \label{fig:juno}
  \centering
  \includegraphics[width=\columnwidth]{figs/juno.pdf}
  \caption{The layout of JUNO and the expected energy spectrum assuming 6 years running. \fixit{still working on this plot, particularly the upper-left insert, it will be replaced with a better spectrum.}. }
\end{figure}

Jiangmen Underground Neutrino Observatory (JUNO), currently under construction, is going to be a multi-purpose experiment with the primary goals to resolve the mass hierarchy issue and precisely measure the neutrino oscillation parameters.
%JUNO also provide great physics opportunities in other topics, e.g, observe neutrinos from supernova, Earth interior and Sun, search atmospheric and sterile neutrinos and perform other exotic searches.
JUNO is located in Kaiping city, Guangdong province, in south of China, about 150 km west of Hong Kong. The JUNO detector is underground with a total vertical overburden of 1800 m.w.e. JUNO observes $\bar\nu_e$ from the Yangjiang nuclear power plant (NPP) and the Taishan NPP at an equal baseline of $\sim$53 km, near the maximal $\theta_{12}$ oscillation. The Yangjiang NPP has six reactors cores of 2.9 GW$_{th}$ thermal power each and the Taishan NPP has planned four cores of 4.6 GW$_{th}$ each, both are under construction. The baseline difference is controlled to less than 500 m to prevent significant degradation of the MH discrimination power~\cite{Li-PRD13}.

The JUNO detectors are being designed. A preliminary design assumes that a central detector is submerged in a water pool with the muon trackers installed on the top of water pool. The water pool is equipped with PMTs and acts as an active muon veto. It also provides passive shielding from the natural radioactivities from the surrounding rock and air. The top muon trackers can provide complementary track measurement of the cosmic muons that miss the central detector. The central detector of JUNO consists of 20 kton of liquid scintillator (LS), contained in a spherical acrylic tank (``Acrylic Sphere" option), or a thin film transparent balloon (``Balloon" option). In the "Acrylic Sphere" option, the acrylic tank is supported by a double layer stainless steel frame. The buffer liquid between the acrylic tank and strut is water, which is physically connected with but optically separated from the outside water Cherenkov detector. In the "Balloon" option, the buffer liquid is non-scintillating linear-alkylbenzene (LAB) contained in a stainless steel sphere of a radius of $\sim $40 m, where the PMTs are mounted. For both options, the PMTs will be protected from implosion by acrylic housing. Taking into account the implosion container and mechanical clearance, the photocathode coverage can reach 75\%-78\% for various options by installing $\sim$18000 PMTs.

Energy resolution is the key to JUNO. An energy resolution of $<3\%/\sqrt{E(MeV)}$ is essential in maintaining the mass hierarchy discrimination power~\cite{Li-PRD13}. Such energy resolution is demonstrated to be achievable by the Monte Carlo simulation, assuming a maximum possible photocathode coverage and a couple of technical improvements: 1) use high quantum efficiency (QE) PMTs with QE $\sim$35\%; 2) improve the LS attenuation length to 20 m (at 430 nm wavelength), which consists of an absorption length of 60 m and a Rayleigh scattering length of 30 m. A new type of 20 inch, high efficiency PMT is being developed by JUNO. Instead of using the traditional dynode, the new PMT design uses micro-channel plate (MCP) to have nearly 4$\pi$ collection of the photoelectrons. Using the super bialkali technique, the quantum efficiency of the photocathode is expected to reach to 35\%. Increasing the light yield and optical transparency of the liquid scintillator is also important for improving the energy resolution. Maximal light yield can be achieved by optimizing the concentration of fluors. To increase the optical transparency of LS, purification of the raw solvent and fluors at the production stage and online purification using Al$_2$O$_3$ column filtration has been found to be effective.

%A position resolution of \fixit{?cm}/$\sqrt{E (MeV)}$ is achievable by using time-based likelihood reconstruction method.
%Other effects like the uncertainties from non-uniformity correction, PMT dark rates, electronics noise, etc, would also affect the energy resolution and should be carefully considered during the detector design.
%The other issue is to reduce the radioactive impurities in LS, particularly for the potential detection of solar neutrinos. An online purification system with distillation is essential to remove U/Th and lead-210 from LS.
%In JUNO, the mass hierarchy information is extracted from the oscillation structure in the energy spectrum.

If the energy non-linearity correction has large uncertainties, particular residual non-linearity shape can fake the oscillation pattern with a wrong mass hierarchy~\cite{Qian-PRD13}. Three main effects can cause non-linear energy response in a LS detector: quenching effect, Cerenkov process and possible electronics non-linear response. The energy scale uncertainty need be controlled within 1\%, this is expected to be achievable in JUNO since it has been achieved in a smaller detector like Daya Bay~\cite{Zhang-Neutrino14}. Interestingly, the multiple $\theta_{13}$ oscillation peaks in the measured $L/E$ spectrum can be utilized to constrain the non-linearity response~\cite{Li-PRD13}.

%With full operation of the reactors from Yangjiang and Taishan NPP, the expected signal at JUNO is 80 reactor $\bar\nu_e$ events per day. The delayed signal is a single 2.2-MeV $\gamma$ released by neutron capture on protons with a capture time of $\sim$200 microseconds. Though the natural radioactivity from detector materials can contaminate the delayed signal, coincidences in energy, position and time can suppress the accidental background to $\sim$2\% of the $\bar\nu_e$ candidates. The cosmogenic $\beta$-n emitters $^9$Li and $^8$He is the other dominant background at JUNO, due to large detector size and relative high cosmic muon rate.

The sources of backgrounds at JUNO are similar to that of KamLAND. However, it's challenging to veto the cosmogenic $^9$Li and $^8$He background at JUNO. It is known that the cosmogenic isotopes are mostly produced by the muons accompanied by large electromagnetic or hadronic showers, and their lateral distance from the parent muon trajectory is roughly exponential. In KamLAND, if a shower muon is tagged, the whole detector will be veto for 2 s. Such veto strategy could lead to significant loss of signal statistics because of the high muon rate at JUNO ($\sim$3 Hz, about $\sim$15 times higher than KamLAND). Thus, the ability of tracking the shower muons in JUNO LS is essential, and it allows a proper cylindrical veto along the muon track to sufficiently reject the $^9$Li and $^8$He with minimal loss of signal statistics.

The sensitivity of mass hierarchy determination at JUNO can exceed 3$\sigma$ after taking into account the realistic locations of the reactor cores, uncertainties of the energy non-linearity, etc. Assuming that the effective mass-squared difference measured by accelerator experiments can reach to $\sim$1.5\% or 1\% precision~\cite{Agarwalla}, then it can help to improve sensitivity to be 3.7$\sigma$ or 4.4$\sigma$\cite{Li-PRD13}. Besides, the precision measurement of the neutrino oscillation parameters allows testing the unitarity of the neutrino mixing matrix~\cite{unitarity13}, which is of great importance. For $\Delta m^2_{21}$, $\Delta m^2_{32}$ and $\sin^2\theta_{12}$, JUNO expects to measure them with a precision better than 1\%. Considering also that the precision of $\sin^2\theta_{13}$ can also be measured to $\sim4\%$ by the on-going $\theta_{13}$ reactor experiments, the unitarity of the first row of the neutrino mixing matrix can be probed to 1\% level.

JUNO provides a unique opportunity to determine MH with precision measurement of reactor neutrino spectrum. Most systematic effects are well understood and studied, although the detector technical challenges are significant. It provides a clean measurement of MH independent of $\theta_{23}$ and unknown CP phase. The reactor measurement will be complementary to the future long-baseline accelerator experiments. The JUNO collaboration currently consists of 48 institutions from 9 countries or regions, with a total number of 329 members. JUNO expects to start data taking in 2020.


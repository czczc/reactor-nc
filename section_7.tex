%!TEX root = reactor_nc_main.tex
%%%%%%%%% Section: Conclusions %%%%%%%%%
\section{Final Remarks} 
\label{sec:prospects}

Over the past $\sim$60 years, nuclear reactors have proven to be one of the most powerful tools to study neutrino oscillations, the quantum-mechanical phenomenon that requires extensions to the Standard Model. Experiments at a few kilometers and at a few hundred kilometers from the reactor cores produced some of the most convincing proofs of neutrino oscillations, by having observed in the $L/E$ domain the oscillatory behavior of reactor $\bar\nu_e$'s during their propagations. Several key parameters governing the neutrino mixing, $\theta_{12}$, $\theta_{13}$, $\Delta{m}^2_{21}$ and $|\Delta{m}^2_{31}|$, were also precisely measured by reactor experiments. 

Nuclear reactors will continue to help us uncover more facts about neutrinos. In the next $\sim$20 years, the upcoming next-generation reactor experiments will tell us what is the neutrino mass hierarchy and whether or not light sterile neutrinos exist. The results will have significant impact on other future programs such as neutrinoless double-beta decay experiments, long-baseline accelerator experiments, astrophysics and cosmology. Ultimately, they may hold the key to our deeper understanding of the fundamental physics and the universe.



%!TEX root = reactor_nc_main.tex
%%%%%%%%% Box 1 %%%%%%%%%
\onecolumngrid
\vspace{12pt}
\fboxsep=12pt

\noindent\fbox{%
    \parbox{0.96\textwidth}{%


{\Large \textcolor{blue}{Box 1: Neutrino Oscillations}}
\vspace{12pt}
\setlength{\parindent}{12pt}

Let us assume that there are two massive neutrinos $\nu_i, i=1,2$ with different masses $m_i$. When they propagate in vacuum over a distance $L$,
each acquires the phase $\nu_i(L) = \nu_i (0) \exp(-i m^2_i L/2E)$. (The overall phase is skipped and it is assumed that the neutrinos are highly relativistic,
$p \sim E - m^2/2E$. Additional phases are acquired when neutrinos propagate in the matter, so-called the ``MSW effect''~\cite{Wolfenstein78,MS85}, which will not be discussed here.) Assume further that the flavor neutrinos $\nu_e$ and $\nu_{\alpha}$, i.e. the neutrinos that are the partners of
charged leptons in the weak interactions, are coherent superpositions of the states $\nu_i$, i.e. $\nu_e = \cos \theta \nu_1 + \sin \theta \nu_2$, and
analogous but orthogonal combination represents the other flavor neutrino $\nu_{\alpha} = -\sin \theta \nu_1 + \cos \theta \nu_2$. This 
mixture is characterized by the parameter $\theta$, so-called mixing angle.  

Consider now a beam of neutrinos that at $L=0$ is pure $\nu_e$. Then
\begin{equation}
\nu_e (L) = \cos \theta e^{-i m_1^2 L/2E} \nu_1(0) + \sin \theta e^{-i m_2^2 L/2E} \nu_2 (0) ~.
\end{equation}
In order to observe this beam at $L$ we must use the weak interactions. Hence we must project the $\nu_i$ back to the flavor basis $\nu_e$ and $\nu_{\alpha}$.
Thus
\begin{equation}
\nu_e (L) = [\cos^2 \theta e^{-i m_1^2 L/2E} + \sin^2 \theta e^{-i m_2^2 L/2E}] \nu_e(0) - 
\sin \theta \cos \theta [e^{-i m_1^2 L/2E} - e^{-i m_2^2 L/2E}] \nu_{\alpha}(0) ~.
\end{equation}
The probability that we detect $\nu_e$ at the distance $L$ is just the square of the corresponding amplitude. The probability
of detecting $\nu_{\alpha}$ is the corresponding square of that amplitude. Therefore, after a bit of a simple algebra,
\begin{equation}
P(\nu_e \rightarrow \nu_e) = 1 - \sin^2 2 \theta \sin^2 \frac{\Delta m^2 L}{4 E} ~,~ P(\nu_e \rightarrow \nu_{\alpha}) =  \sin^2 2 \theta \sin^2 \frac{\Delta m^2 L}{4 E} ~,
\end{equation}
where $\Delta m^2 = m^2_2 - m^2_1$ is the difference of the squares of the neutrino masses. 

Thus, we see that, provided $\Delta m^2 \ne 0$ and if $\theta \ne 0$ or $\pi/2$, the composition of the neutrino beam oscillates as a function of $L/E_{\nu}$ 
with the amplitude $\sin^2 2 \theta$ and wavelength
\begin{equation}
L_{osc} = 4 \pi \frac{E}{\Delta m^2} ~~\equiv ~~ L_{osc} ({\rm m}) = \frac{2.48 E {\rm (MeV)}}{ \Delta m^2 {\rm (eV^2)}} ~. 
\end{equation}

Observation of neutrino oscillations, therefore, constitutes a proof that at least some of the neutrinos have a finite mass and that the superposition is
a nontrivial one. Generalization to the realistic case of three neutrino flavors and three
states of definite mass is straightforward. The corresponding mixing is then characterized by three mixing angles $\theta_{12}, \theta_{13}, \theta_{23}$,
 one possible $CP$ violating phase $\delta_{CP}$,
and two mass square differences $\Delta m^2_{21}$ and  $\Delta m^2_{31}$. The third mass square difference 
$\Delta m^2_{32} = m^2_3 - m^2_2 \equiv \Delta m^2_{31} - \Delta m^2_{21}$ is simply related
to those two. 

Neutrino oscillations have no classical analog. They are purely quantum-mechanical phenomenon, consequence of the coherence of the
superposition of states.

    }%
}

\vspace{10pt}

\twocolumngrid


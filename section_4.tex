%!TEX root = reactor_nc_main.tex
%%%%%%%%% Section: Theta_13 Experiments %%%%%%%%%
\section{Searching for the Smallest Oscillation Angle: $\theta_{13}$ Experiments} 
\label{sec:theta13}
In contrast to the CKM matrix in quark mixing, where all three mixing angles are very small, the mixing angles in the neutrino mixing matrix appear to be large: $\theta_{23}$, measured by the atmospheric and accelerator neutrino experiments, is about $45^\circ$, and $\theta_{12}$, measured by the solar neutrino experiments and KamLAND, is about $33^\circ$. It was therefore natural to expect the third mixing angle, $\theta_{13}$, to be of similar magnitude.

The cleanest way to measure $\theta_{13}$ is through kilometer-baseline reactor neutrino oscillation experiments. The reactor $\bar\nu_e$ oscillation at $\sim$km is dominated by the $\theta_{13}$ terms. Unlike accelerator neutrino experiments, the reactor measurements are independent of CP-violating phase and $\theta_{23}$, and only slightly dependent of the neutrino mass hierarchy and matter effect. A high precision measurement can therefore be achieved. 

In the 1990s, two first-generation kilometer-baseline reactor experiments, CHOOZ \cite{Chooz} and PALO VERDE \cite{Paloverde} were built to measure $\theta_{13}$. The CHOOZ detector was built at a distance of $\sim$1050 m from the two reactors of the CHOOZ power plant of \'{E}lectricite\'{d}e France in the Ardennes region of France. It took data from April 1997 until July 1998. The PALO VERDE detector was built at distances of 750, 890 and 890 m from the three reactors of the Palo Verde Nuclear Generating Station in the Arizona desert of the United States. It took data between October 1998 and July 2000. Surprisingly, neither experiment was able to observe the $\bar\nu_e$ deficit caused by $\theta_{13}$ oscillation. As a result, only an upper limit of $\sin^22\theta_{13} < 0.15$ at 90\% C.L. was obtained.

The results from CHOOZ and PALO VERDE motivated many phenomenological speculations of neutrino mixing patterns such as bimaximal and tribimaximal mixing \cite{Harrison,Altarelli}. In most of these theories, $\theta_{13}$ is either zero or very small. A direct consequence of a vanishing $\theta_{13}$ is that the CP violation in the leptonic sector, even if large, can never be observed in the neutrino oscillation experiments. The importance of knowing the precise value of $\theta_{13}$ provoked a series of world-wide second-generation kilometer-baseline reactor experiments in the 21st century, including Double Chooz \cite{DChooz} in France, RENO \cite{Reno} in Korea and Daya Bay \cite{Dayabay} in China to push the sensitivity to $\theta_{13}$ to below $10^\circ$. Table \ref{tab:theta13} summarizes some of the key parameters of the five aforementioned experiments.

\begin{table}[!htb]
  \begin{tabular}{lcccc}
  \hline
  & Power & Baseline & Mass & Overburden \\
  & (GW$_{th}$) & (m) & (ton) & (m.w.e)    \\
  \hline
  CHOOZ        & 8.5  & 1050  & 5    & 300\\
  PALO VERDE   & 11.6 & 750--890  & 12  & 32\\
  \hline
  Double CHOOZ & 8.5  & 400  & 8  & 120\\
               &      & 1050 & 8  & 300\\
  RENO         & 16.8 & 290  & 16   & 190\\
               &      & 1380 & 16   & 540\\
  Daya Bay     & 17.4 & 360  & 2$\times$20   & 250\\
               &      & 500  & 2$\times$20   & 265\\
               &      & 1580 & 4$\times$20   & 860\\
  % 1998\tablenote{predicted} & 200 & 300 & 1500  & 2000\\
  \hline
  \end{tabular}
  \caption{Summary of key parameters of the reactor $\theta_{13}$ experiments, including reactor thermal power (in giga-watts), distance to reactors, detector target mass and overburden of the underground site (in meter-water-equivalent).}
\label{tab:theta13}
\end{table}

A common technology used in both the first and second generation experiments is the gadolinium-loaded liquid scintillator as $\bar\nu_{e}$ detection target. Gd has a high thermal neutron capture cross section. With $\sim$0.1\% gadolinium loading, the neutron capture time is reduced to $\sim$28 seconds from $\sim$200 seconds for the un-loaded scintillator. Furthermore, Gd deexcitation after the capture releases an 8-MeV gamma-ray cascade, which gives a delayed signal well above natural radioactivity (In contrast, neutron capture on protons results in a single 2.2-MeV $\gamma$.) The accidental coincidence background is therefore drastically reduced.

The most significant improvement of the second-generation experiments over the previous ones is the addition of near detectors at baselines of a few hundred meters. As discussed in Section II, the uncertainty in the reactor antineutrino flux prediction is large (2--5\%.) This flux uncertainty, however, can be canceled out from a relative measurement between near and far detectors. The Double Chooz experiment expands CHOOZ by adding a near detector at a distance of $\sim$400 m. The installation of the near detector, however, was delayed due to civil construction. Double Chooz started taking data in May 2011 with only a far detector, and uses the Bugey-4 measurements to normalize the reactor flux. The RENO experiment is built near the six reactors of the Yonggwang nuclear power plant in Korea. The two identical detectors are located at 290 and 1380 m, respectively, from the center of reactor array. RENO started taking data in August 2011. The Daya Bay experiment is built near the six reactors of the Daya Bay nuclear power plant in southern China. Daya Bay has eight identical antineutrino detectors (ADs). Two ADs are placed at $\sim$360 m from the Daya Bay reactor clusters. Two ADs are placed at $\sim$500 m from the Lingao reactor clusters. And four ADs are placed at a far site $\sim$1580 m away from the reactor complex. This modular detector design further allows Daya Bay to largely cancel out the detector-correlated systematic uncertainties. Daya Bay started taking data in December 2011.

Compared to the first-generation experiments, the second-generation experiments have much larger statistics by utilizing higher power reactors and larger detectors. Among them, Daya Bay has the largest reactor thermal power (17.4 GW$_{th}$) and target mass (80 tons at the far site,) as shown in Table \ref{tab:theta13}. The underground sites are much deeper to allow better shielding from cosmogenic background, in particular compared to the case of PALO VERDE. Better chemical recipes of gadolinium-loaded liquid scintillator also improves the overall detector performance and long term stability.

The second-generation reactor experiments were a huge success. In 2012, all three experiments, Double Chooz, Daya Bay and RENO, reported clear evidences of $\bar\nu_{e}$ disappearance \cite{DChooz,Reno,Dayabay} at $\sim$kilometer baselines with only a few month's running. In particular, Daya Bay excluded $\theta_{13}=0$ by 5.2 standard deviation \cite{Dayabay}. The precision of the measurement of $\theta_{13}$ improves quickly with more data.   Fig.~\ref{fig:dayabay} (left) shows the ratio of the measured versus expected $\bar\nu_{e}$ rate, assuming no oscillation, at the 8 detectors of the Daya Bay experiment, with the data collected through November 2013 \cite{Zhang-Neutrino14}. The $\sim$6\% deficit at the far site relative to the near sites is clearly visible and fits nicely to the theoretical oscillation curve (in red) with $\sin^22\theta_{13} = 0.084 \pm 0.005$. Although last known, the precision in $\theta_{13}$ measurement (6\%) is now the best among all three mixing angles.

Similar to KamLAND, the ratio of the detected $\bar\nu_{e}$ events to no-oscillation expectation at Daya Bay is plotted in Fig.~\ref{fig:dayabay} (right) as a function of $L/E$. The combination of the three experimental halls shows a near-complete cycle of the expected periodic oscillation feature. The smaller amplitude and shorter period of the oscillation, compared to the case of KamLAND, indicates the different oscillation component driven by $\theta_{13}$ and $\Delta{m}^2_{31}$. The best-fit frequency of the oscillation yields $|\Delta{m}^2_{31}| = 2.47^{+0.11}_{-0.10} \times 10^{-3}$ eV$^2$ (assuming normal mass hierarchy), which is in good agreement and of comparable precision with the results from atmospheric and accelerator neutrino experiments. 

The longstanding puzzle of the value of $\theta_{13}$ is now successfully resolved. The relatively large value of $\theta_{13}$ opens the gateway for future experiments to determine the neutrino mass hierarchy and measure the CP violating phase in the leptonic sector.

\begin{figure*}[htb] \label{fig:dayabay}
  \centering
  \includegraphics[width=0.95\textwidth]{figs/dayabay.pdf}
  \caption{Daya Bay results: (left) Ratio of the detected versus expected $\bar\nu_{e}$ signals at the 8 antineutrino detectors (ADs) located in three experimental halls.  The oscillation survival probability at the best-fit value is given by the red curve.
  (right) Ratio of the background-subtracted $\bar\nu_e$ spectrum to the expectation for no-oscillation in the three experimental halls, re-expressed as a function of $L_{\textrm{eff}}/E_{\nu}$. The effective baseline $L_{\textrm{eff}}$ is determined for each experimental hall (EH) to an effective oscillated flux from a single baseline. The oscillation survival probability using the best estimates of $\theta_{13}$ and $|\Delta{m}^2_{31}|$ is given by the red curve.}
\end{figure*}






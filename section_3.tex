%!TEX root = reactor_nc_main.tex
%%%%%%%%% Section: KamLAND %%%%%%%%%
\section{Exploring Solar Neutrino Oscillations on Earth: KamLAND} 
\label{sec:kamland}

Since the late 1960s, a series of solar neutrino experiments (Homestake \cite{Homestake}, GALLEX \cite{GALLEX}, SAGE \cite{SAGE}, Kamiokande \cite{Kamiokande}, Super-Kamiokande \cite{Super-Kamiokande}) have observed a large deficit of solar $\nu_e$ flux with respect to the Standard Solar Model (SSM) \cite{Bahcall} prediction. It appeared that more than half of the solar neutrinos were missing. This was referred to as the ``Solar Neutrino Problem''. In 2001, the SNO experiment \cite{SNO} successfully measured the total flux of all three neutrino flavors through the neutral current channel $\nu + d \to \nu + p + n$ using heavy water as a target, and yielded consistent results with the SSM. The SNO result is the ``smoking gun'' evidence of neutrino oscillation explanation to the Solar Neutrino Problem: The solar neutrinos, produced as $\nu_e$ from fusion inside the Sun, have transformed into other flavors when they arrive at the Earth.

The solar neutrino experiments left several possible solutions in the oscillation parameter space of $\theta_{12}$ and $\Delta m^2_{21}$. 
A precise measurement of these parameters and demonstration of the oscillatory feature, however, is hindered by the relatively large uncertainties in the $\nu_{e}$ flux predicted by the SSM, the large matter effect inside the Sun, and the extremely long distance the neutrinos travel. 
A reactor neutrino experiment, measuring the same disappearance channel as the solar neutrino experiments assuming $CPT$ invariance, overcomes these difficulties. 
With well understood and controllable $\sim$MeV $\bar\nu_e$ source, a reactor experiment at $\sim$100 km baselines can explore with high precision the so-called ``Large Mixing Angle (LMA)'' parameter region suggested by the solar neutrino experiments. To do that, the KamLAND experiment \cite{Kamland03} was built in early 2000s to explore the solar neutrino oscillations on Earth.

In order to shield against the cosmic rays, the KamLAND detector was placed at the site of the former Kamiokande experiment \cite{Kamiokande} under the summit of Mt.~Ikenoyama in the Japanese Alps. The vertical overburden is 2700 meter-water-equivalent (m.w.e). It is surrounded by 55 Japanese nuclear reactor cores, which then produced about 30\% of the total electricity in Japan. The $\bar\nu_e$ flux weighted baseline is about 180 km, well suited to study the parameters suggested by the solar neutrino experiments. The reactor operation information such as thermal power, fuel burn-up, and fuel exchange and enrichment records are provided by all Japanese reactors, which allows KamLAND to calculate the instantaneous fission rate of each isotope accurately. 

The KamLAND detector consists of 1 kton of highly purified liquid scintillator (LS), enclosed in a 13-m-diameter transparent balloon suspended by ropes in mineral oil (MO). The MO is housed inside a 18-m-diameter stainless steel (SS) sphere, where an array of 1879 20-inch photomultiplier tubes (PMTs) is mounted. The MO shields the inner LS region from external radiation from PMTs and SS. 3.2 kton of purified water is used to provide further shielding against ambient radiation and operates as an active cosmic muon veto detector. With regular central-axis deployments of radioactive sources and dedicated off-axis deployments, KamLAND achieved a position resolution of $12$ cm$/\sqrt{E(\textrm{MeV})}$, energy resolution of $6.5\%/\sqrt{E(\textrm{MeV})}$ and absolute energy-scale uncertainty of 1.4\%.

Even with such powerful reactor $\bar\nu_e$ sources and a large detector, the long baseline suppresses the expected signal at KamLAND to only about one reactor $\bar\nu_e$ event per day. The experiment is only possible thanks to the powerful coincidence signature (the positron followed by the delayed neutron capture $\gamma$) of inverse beta decay. A time difference of less than 1 millisecond and distance less than 2 meters between the prompt and delayed events is required in the analysis. Only the innermost 6-m radius scintillator region is used to reduce the accidental coincidence from external gamma-rays. Information about the event energy, position and time were used to further reduce the accidental background to $\sim$5\% of the candidates.

The other dominant background ($\sim$10\%) at KamLAND is caused by the $^{13}$C$(\alpha,n)^{16}$O reaction where the $\alpha$-decay comes from $^{210}$Po, a daughter of $^{222}$Rn introduced into the scintillator during construction. The neutron scattering off proton or $^{16}$O$^*$ de-excitation produces a prompt signal, followed by a neutron capture delayed signal, which mimics a true $\bar\nu_e$ event. The rest of the backgrounds include: the antineutrinos produced in the decay chains of $^{232}$Th and $^{238}$U in the Earth's interior (geoneutrinos); the cosmogenic beta delayed-neutron emitters $^{9}$Li and $^{8}$He;  the fast neutrons from muons passing through the surrounding rock, as well atmospheric neutrinos. 

\begin{figure*}[htb] \label{fig:kamland}
  \centering
  \includegraphics[width=0.9\textwidth]{figs/kamland.pdf}
  \caption{KamLAND results: (left) Prompt energy spectrum of $\bar\nu_e$ candidate
  events. All histograms corresponding to reactor spectra and expected
  backgrounds incorporate the energy-dependent selection efficiency
  (top panel). The shaded background and geoneutrino histograms are
  cumulative. 
  % Statistical uncertainties are shown for the data; the band on the blue histogram indicates the event rate systematic uncertainty. 
  (right) Ratio of the background-subtracted
  $\bar\nu_e$ spectrum to the expectation for no-oscillation as a
  function of $L_{0}/E$. $L_{0}$ is the effective baseline taken as a
  flux-weighted average ($L_{0}$\,=\,180\,km). The oscillation survival probability using the best estimates of $\theta_{12}$ and $|\Delta{m}^2_{21}|$ is given by the blue curve.}
\end{figure*}

Fig.~\ref{fig:kamland} (left) shows the prompt energy spectrum of $\bar\nu_e$ candidate
events, observed with 2.9 kton$\cdot$year exposure, overlaid with the expected reactor $\bar\nu_{e}$ and background spectra. A total of 1609 events were observed, which is only about 60\% of the expected signal if there is no oscillation. The ratio of the background-subtracted $\bar\nu_e$ candidate events to no-oscillation expectation is plotted in Fig.~\ref{fig:kamland} (right) as a function of L$_0$/E. The spectrum indicates almost two cycles of the periodic feature as expected from neutrino oscillation, disfavoring other explanations of the $\bar\nu_e$ disappearance.

The KamLAND results \cite{Kamland03,Kamland05,Kamland08} are highly consistent with the solar neutrino experiments, and pinned down the solar neutrino oscillation solution to the LMA region. When combined with the results from SNO, it yields the most precise measurements of $\tan^2\theta_{12} = 0.47^{+0.06}_{-0.05}$ and $\Delta m^2_{21} = 7.59^{+0.21}_{-0.21} \times 10^{-5}$ eV$^2$. It marks the beginning of a precision era in the neutrino oscillation experiments.


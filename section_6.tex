%!TEX root = reactor_nc_main.tex
%%%%%%%%% Section: Sterile Neutrinos: Very Short Baseline %%%%%%%%%
\section{Searching for Sterile Neutrinos: Very Short Baseline Reactor Experiments}

The number of light neutrino species that determined from the invisible decay width of the Z boson by combining all the experiments in the 1990s at the electron-position colliders SLC and LEP~\cite{EW-2005} is in agreement with the three observed flavors of neutrino. Whether there exist extra species of neutrinos with no ordinary weak interactions, so-called sterile neutrinos, is one of the fundamental questions in neutrino physics and cosmology.

In the past decade, the picture of neutrino oscillations has been established and the mass squared difference $\delta m^2$ can be measured in different oscillation channels. The solar-neutrino~\cite{SNO} and the reactor-antineutrino~\cite{Kamland03} experiment has observed the neutrino oscillation in $\nu_e$ and $\bar\nu_e$ disappearance channel at $\delta m^2\sim8\times10^{-5}$eV$^2$. The atmospheric-neutrino~\cite{SuperK98} and long-baseline accelerator-neutrino experiments~\cite{K2K-2003,MINOS06} has observed the neutrino oscillation in $\nu_\mu$ disappearance channel at $\delta m^2\sim3\times10^{-3}$eV$^2$. In the early 2000s, the LSND experiment~\cite{LSND2001} reported anomalous event excesses in the $\bar\nu_\mu\rightarrow\bar\nu_e$ appearance channel, and interpreted it as an evidence for $\bar\nu_\mu\rightarrow\bar\nu_e$ oscillations at the $\delta m^2\sim1$eV$^2$ scale. However, such excess was not confirmed by a similar experiment KARMEN~\cite{KARMEN2002}.

These three different $\delta m^2$ scales do not agree with the Standard Model which requires three neutrino species. The phenomenological models introducing one or more additional ``light" sterile neutrinos (due to their small masses $\lesssim$1 eV)~\cite{Serel04} could explain the tension. However, the light sterile neutrinos are less motivated, comparing to the heavy sterile neutrinos that can provide an elegant interpretation of the small neutrino masses~\cite{SeeSaw} and contribute to the mechanism for matter-antimatter asymmetry of the universe~\cite{leptogenesis}. Thus the tension should be addressed experimentally. The MiniBooNE experiment~\cite{MiniBooNE2007} was motivated to check the results from LSND, and its recent results did not refute LSND~\cite{MiniBooNE2013}. Interestingly, most recent MINOS data has shown no evidence for $\nu_\mu$ disappearance into sterile neutrinos at $\delta m^2<$1 eV$^2$~\cite{Sousa-Neutrino14}. Furthermore, future short-baseline accelerator-neutrino experiments~\cite{nuSTORM,IsoDAR} are being proposed to solve the sterile neutrino issue.

%Sterile neutrino~\cite{sterileWP} have been an active field in the past decade, the fundamental issue is that

%Neutrino mixing is usually incorporated in a framework in which three-flavor neutrino states, $\nu_e$, $\nu_\mu$ and $\nu_\tau$, are superpositions of the three mass eigenstates $\nu_i$. However, this framework might be incomplete as indicated by several recent experimental results, including the anomalous event excesses in $\bar\nu_e$ appearance experiments (LSND~\cite{LSND2001} and MiniBooNE~\cite{MiniBooNE2013}),

The $\nu_e$ rate deficits have been found in the GALLEX~\cite{GALLEX,GALLEX2010} and the SAGE~\cite{SAGE,SAGE2009} solar neutrino detectors with intense artificial radioactive sources, often referred as ``Gallium anomaly". This could be explained by the oscillation caused by sterile neutrino~\cite{Guinti2010}, although the significance is about $\sim2\sigma$. To unambiguously clarify this anomaly, new experiments are proposed, using a very intense $^{51}$Cr or $^{44}$Ce-$^{144}$Pr source~\cite{Cribier2011,Dwyer2013,SOX,CeLAND} next to a large LS detector.

%In addition, most recent cosmology surveys (e.g, WAMP~\cite{WMAP2011}, PLANCK~\cite{PLANCK2013,PLANCK-BICEP2}) also showed hits of more than three effective number of neutrino species.

The hint of sterile neutrinos also exists in the reactor neutrino experiments. Recently, re-evaluations of the reactor $\bar\nu_e$ flux have showed an increase in the predicted $\bar\nu_e$ rate~\cite{Mueller2011, Huber2011}. Combining with the reactor experimental data at baselines between 10-100m, those calculations suggest a $\sim(5.7\pm2.3)$\% deviation between the measured and predicted reactor $\bar\nu_e$ flux~\cite{Mention2011}, so-called ``reactor anomaly". This reactor anomaly could be a hint of existing additional sterile neutrino states with mass splitting of $\sim1$ eV$^2$~\cite{Guinti2011}.

However, the reactor anomaly might be caused by imperfect predictions of reactor $\bar\nu_e$ flux. In a later analysis~\cite{Zhang13}, by including the absolute reactor $\bar\nu_e$ flux results from PALO VERDE, CHOOZ and Double CHOOZ at km-scale baselines and using the measured $\theta_{13}$ value from Daya Bay, the new world average ratio of the measured flux to the prediction is only $\sim1.4\sigma$ lower than the unity, thus the significance of the reactor anomaly is weakened. In the past year the uncertainties in the analysis of the reactor anomaly have been revisited, and it has been found that the components of the aggregate fission spectra containing $\sim30\%$ forbidden decays introduce $\sim$4\% uncertainty in the predicted shape of the reactor $\bar\nu_e$ flux~\cite{Hayes}. This suggests that the ultimate solution to reactor anomaly requires a more precise direct measurement of the antineutrino flux. Most recently, the sterile neutrino search at Daya Bay using only the relative spectral distortion~\cite{DayaBaySterile} showed no evidence in the $10^{-3}$ eV$^2<\delta m^2<0.3$ eV$^2$ range.

Very short baseline reactor experiments have been strongly motivated by the reactor anomaly and the sterile neutrino hypothesis~\cite{sterileWP}, and been proposed worldwide in U.S (PROSPECT~\cite{PROSPECT}), Europe (NUCIFER~\cite{NUCIFER-2010, NUCIFER-2014}, STEREO~\cite{NUCIFER-2014}, DANSS~\cite{DANSS}, NEUTRINO-4~\cite{NEUTRINO4-2012,NEUTRINO4-2014}, POSEIDON~\cite{POSEIDON}, SOLID~\cite{SoLid}) and Korea (HANARO~\cite{HANARO}). Each proposed experiment is located at a research reactors with compact, high $^{235}$U enrichment (HUE) core. The detector in each experimental proposal is deployed at a distance of 4-20 m from the reactor core, with a shallow depth of $\sim$10 m.w.e level. Table ~\ref{tab:sterile} summarizes some of the key parameters of the worldwide proposed very short baseline reactor experiments.

\begin{table}[t]
  \begin{tabular}{lcccc}
  \hline
  & Power & Baseline & M$_{target}$ & Overburden \\
  & (MW$_{th}$) & (m) & (ton) & (m.w.e)    \\
  \hline
  PROSPECT  & 85  & 4-20 & 1 \& 10     & $<$10 \\
%  \hline
  NUCIFER   & 70 & $\sim$7  & 0.7 & 13\\
  STEREO & 57  & $\sim$10 & 1.75  & 15\\
  DANSS & 3000  & 9.7-12.2  & 0.9  & 50\\
  NEUTRINO-4 & 100  & 6-12  & 1.5  & $\sim$10 \\
  POSEIDON & 100  & 5-8  & $\sim1.3$ & few\\
  SOLID & 45-80 & 6.8  & 2.9  & 10\\
%  \hline
  HANARO & 30  & 6  & $\sim$1  & $<$10\\
  \hline
  \end{tabular}
  \caption{Summary of key parameters of the proposed very short-baseline reactor experiments, including reactor thermal power (in giga-watts), distance to reactors, detector target mass and overburden of the underground site (in meter-water-equivalent).}
\label{tab:sterile}
\end{table}

Good resolution of position and energy in the $\bar\nu_e$ detector is important for the precision oscillation measurement at a meter-scale short baseline. Challenges exists in mitigating the cosmogenic fast neutrons and the reactor-related high energy gammas backgrounds, because of the shallow depth and the short-baseline. Given the tight experimental space near reactor core, sufficient active muon veto and passive shielding structure have to be carefully designed. Highly segmented detectors along the axis pointing toward the core are preferred~\cite{PROSPECT,DANSS,SoLid,HANARO}. The advantages include good position resolution and the ability of background rejection by tagging multi-site event topologies. However, it's challenging to perform calibration and control relative variation of each segmented detector. For the detectors without segmentation~\cite{NUCIFER-2010,NEUTRINO4-2014,POSEIDON} or poor segmentation~\cite{NUCIFER-2014}, the capability of pulse-shape discriminating (PSD) in the gadolinium-loaded LS is important for background rejection. In addition, neutron capture on $^6$Li results in an alpha and a tritium, which can give very discriminant neutron signal. Thus the PROSPECT experiment is developing $^6$Li-loaded scintillator (Li-LS), while the SOLID experiment will use a composite solid scintillator consists of $^6$LiF-ZnS as neutron layer and plastic scintillator.

Very short-baseline reactor experiments have unique signature of new oscillation pattern in a relative large range of energy (1-8 MeV) and baseline (4-20m), which allows good sensitivity to test reactor anomaly and sterile neutrinos with $\delta m^2\sim1$eV$^2$. There are worldwide interests and several experiments are being proposed or under construction. In 2015-16, a few of them will start data taking~\cite{Lhuillier-Neutrino14} and expect to test the reactor anomaly with a sensitivity better than 5$\sigma$ after 1-3 years running.



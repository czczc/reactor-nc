%!TEX root = reactor_nc_main.tex
%%%%%%%%% Section: Sterile Neutrinos: Very Short Baseline %%%%%%%%%
\section{Searching for Sterile Neutrinos: Very Short Baseline Reactor Experiments}

The number of light active neutrino flavors was determined from the precision electroweak measurements of the decay width of the Z boson to be three \cite{EW-2005}. 
Such a three-neutrino framework has been extremely successful in explaining the measurements from neutrino oscillation experiments. 
Namely, only two oscillation frequencies, corresponding to the two mass-squared differences ($\Delta m_{21}^2\sim7.6\times10^{-5}$ eV$^2$ and $|\Delta m_{31}^2|\sim2.4\times10^{-3}$ eV$^2$), were observed by the solar, atmospheric, accelerator and reactor neutrino oscillation experiments. 
However, in the early 2000s, the LSND experiment~\cite{LSND2001} reported anomalous event excesses in the $\bar\nu_\mu\rightarrow\bar\nu_e$ appearance channel, which could be interpreted as an oscillation at $\Delta m^2\sim1$ eV$^2$ scale. 
The LSND result contradicted the three-neutrino framework in the Standard Model, thus was often referred to as the ``LSND Anomaly''.

The ``LSND Anomaly'', therefore, could indicate the existence of additional
fourth or more neutrino families. 
Since they have light masses (m $\sim$ 1 eV) but do not couple to the Z bosons, they must lack weak interactions and are therefore called sterile neutrinos. 
They are observable only through their sub-dominant mixing with the familiar active neutrinos. 
The light sterile neutrinos, coincidentally, are also among the leading candidates to resolve outstanding puzzles in astrophysics and cosmology~\textcolor{red}{(add refs.)}.
On the other hand, the light sterile neutrinos are generally not natural in the theories that extend the neutrino Standard Model. 
For example, the popular type-I see-saw model \cite{SeeSaw}, which provides an elegant explanation to the small neutrino masses and the matter-antimatter asymmetry of the universe \cite{leptogenesis}, only naturally predicts very heavy sterile neutrinos (m $>10^{18}$ eV).
If the light sterile neutrinos indeed exist as LSND indicates, they would suggest new frontiers in both experimental and theoretical physics.

The ``LSND Anomaly'' is so-far still experimentally unresolved despite the many efforts to do so. 
There are several hints supporting LSND's findings, but are not conclusive.
The MiniBooNE experiment, designed at a similar $L/E$ baseline as LSND using accelerator neutrinos, observed similar event excesses in the $\nu_{\mu}\rightarrow\nu_e$ and $\bar\nu_{\mu}\rightarrow\bar\nu_e$ appearance channels \cite{MiniBooNE2013}. 
The GALLEX~\cite{GALLEX2010} and the SAGE~\cite{SAGE2009} solar neutrino experiments, during their calibrations using intense neutrino sources ($^{51}$Cr, $^{37}$Ar), observed a $\sim$24\% deficit in the $\nu_e$ disappearance channel. 
This deficit is often referred to as the ``Gallium Anomaly''. 
Recently, re-evaluations of the reactor $\bar\nu_e$ flux resulted in an increase in the predicted $\bar\nu_e$ rate~\cite{Mueller, Huber}. 
Combining the new predictions with the reactor experimental data at baselines between 10-100m suggests a $\sim$4--6\% deficit between the measured and predicted reactor $\bar\nu_e$ flux, so-called the ``Reactor Antineutrino Anomaly"~\cite{Mention2011,Zhang13}. 
These experimental anomalies can be interpreted by additional light sterile neutrinos~\cite{Guinti2011}, but might also be caused by imperfect knowledge of the theoretical predictions or experimental systematics.  
The preferred region ($\Delta{m}^2\sim1$ eV$^2$ and $\sin^22\theta\sim0.01$), however, is in tension~\textcolor{red}{(add refs.)} with the limits derived from other appearance~\textcolor{red}{(add refs.)} or disappearance searches~\textcolor{red}{(add refs.)}, in particular with the recent sterile neutrino search results from MINOS+~\cite{Sousa-Neutrino14} and Daya Bay~\cite{DayaBaySterile}.

Due to the strong motivations but rather confusing present experimental status, searching for light sterile neutrinos is a prioritized program~\cite{sterileWP} with many proposed next-generation neutrino oscillation experiments.
Different technologies will be used, including short-baseline accelerator experiments~\cite{nuSTORM,IsoDAR}~\textcolor{red}{(add refs for LAr experiments)}, $^{51}$Cr ($^{44}$Ce-$^{144}$Pr) $\nu_e$ ($\bar\nu_e$) source experiments~\cite{Cribier2011,Dwyer2013,SOX,CeLAND}, as well as very short baseline ($<10$ m) reactor (VSBR) $\bar\nu_e$ experiments. 
In order to unambiguously resolve the ``LSND Anomaly'', the oscillation cycles in the $L/E$ space need to be observed, similar as in KamLAND (Fig.~\ref{fig:kamland}) and Daya Bay (Fig.~\ref{fig:dayabay}). VSBR experiments provide unique opportunities given the many advantages provided by reactors.

Multiple VSBR experiments have been proposed globally in U.S (PROSPECT~\cite{PROSPECT}), Europe (NUCIFER~\cite{NUCIFER-2010, NUCIFER-2014}, STEREO~\cite{NUCIFER-2014}, DANSS~\cite{DANSS}, NEUTRINO-4~\cite{NEUTRINO4-2012,NEUTRINO4-2014}, POSEIDON~\cite{POSEIDON}, SOLID~\cite{SoLid}) and Korea (HANARO~\cite{HANARO}). Each of the proposed experiments is located at a research reactors with a compact, high $^{235}$U enrichment (HUE) core. The detector in these proposals is deployed at a distance of 4-20 m from the reactor core, with a shallow depth of $\sim$10 m.w.e level. Table ~\ref{tab:sterile} summarizes some of the key parameters of the worldwide proposed very short baseline reactor experiments.

\begin{table}[t]
  \begin{tabular}{lcccc}
  \hline
  & Power & Baseline & Mass & Overburden \\
  & (MW$_{th}$) & (m) & (ton) & (m.w.e)    \\
  \hline
  PROSPECT  & 85  & 4-20 & 1 \& 10     & $<$10 \\
%  \hline
  NUCIFER   & 70 & $\sim$7  & 0.7 & 13\\
  STEREO & 57  & $\sim$10 & 1.75  & 15\\
  DANSS & 3000  & 9.7-12.2  & 0.9  & 50\\
  NEUTRINO-4 & 100  & 6-12  & 1.5  & $\sim$10 \\
  POSEIDON & 100  & 5-8  & $\sim1.3$ & few\\
  SOLID & 45-80 & 6.8  & 2.9  & 10\\
%  \hline
  HANARO & 30  & 6  & $\sim$1  & $<$10\\
  \hline
  \end{tabular}
  \caption{Summary of key parameters of the proposed very short baseline reactor experiments, including reactor thermal power (in Mega-watts), distance to reactors, detector target mass and overburden of the underground site (in meter-water-equivalent).}
\label{tab:sterile}
\end{table}

Good resolution of position and energy in the $\bar\nu_e$ detector is 
%important 
crucial for the precision oscillation measurement at a meter-scale short baseline. Challenges 
%exists 
are in mitigating the cosmogenic fast neutrons and the reactor-related high energy gammas backgrounds, because of the shallow depth and the short-baseline. Given the tight experimental space near the reactor core, sufficient active muon veto and passive shielding structure have to be carefully designed. Highly segmented detectors along the axis pointing toward the core are preferred~\cite{PROSPECT,DANSS,SoLid,HANARO}. The advantages include good position resolution and the ability of background rejection by tagging multi-site event topologies. However, it is challenging to perform calibration and control relative variation of each segment of the detector. For the detectors without segmentation~\cite{NUCIFER-2010,NEUTRINO4-2014,POSEIDON} or poor segmentation~\cite{NUCIFER-2014}, the capability of pulse-shape discriminating (PSD) in the gadolinium-loaded LS is important for background rejection. In addition, neutron capture on $^6$Li results in an alpha and a tritium, which can give very discriminant neutron signal. Thus the PROSPECT experiment is developing $^6$Li-loaded scintillator (Li-LS), while the SOLID experiment will use a composite solid scintillator consists of $^6$LiF-ZnS as neutron layer and plastic scintillator.

Very short-baseline reactor experiments have unique signature of new oscillation pattern in a relatively large range of energy (1-8 MeV) and baselines (4-20m), which allows a good sensitivity to test reactor anomaly and sterile neutrinos with $\Delta m^2\sim1$eV$^2$. There are worldwide interests and several experiments are being 
%proposed 
actively considered or under construction as shown in Table \ref{tab:sterile}. In 2015-16, a few of them will begin data taking~\cite{Lhuillier-Neutrino14} and expect to test the reactor anomaly with a sensitivity better than 5$\sigma$ after 1-3 years of running.


%!TEX root = reactor_nc_main.tex
%%%%%%%%% Section: Sterile Neutrinos: Very Short Baseline %%%%%%%%%
\section{Searching for Sterile Neutrinos: Very Short Baseline Reactor Experiments}

The number of light active neutrino flavors was determined from the precision electroweak measurements of the decay width of the Z boson to be three~\cite{EW-2005}. 
Such a three-neutrino framework has been extremely successful in explaining the measurements from neutrino oscillation experiments. 
Namely, only two oscillation frequencies, corresponding to the two mass-squared differences ($\Delta m_{21}^2\sim7.6\times10^{-5}$ eV$^2$ and $|\Delta m_{31}^2|\sim2.4\times10^{-3}$ eV$^2$), were observed by the solar, atmospheric, accelerator and reactor neutrino oscillation experiments. 
However, in the early 2000s, the LSND experiment~\cite{LSND2001} reported anomalous event excesses in the $\bar\nu_\mu\rightarrow\bar\nu_e$ appearance channel, which could be interpreted as an oscillation at $\Delta m^2\sim1$ eV$^2$ scale. 
The LSND result contradicted the three-neutrino framework in the Standard Model, thus was often referred to as the ``LSND anomaly''.

The LSND anomaly, therefore, could indicate the existence of additional
fourth or more neutrino families. 
Since they have light masses (m $\sim$ 1 eV) but do not couple to the Z bosons, they must lack weak interactions and are therefore called sterile neutrinos. 
They are observable only through their sub-dominant mixing with the familiar active neutrinos. 
The light sterile neutrinos, coincidentally, are also among the leading candidates to resolve outstanding puzzles in astrophysics and cosmology~\cite{Dodelson,Kusenko,Wyman,Battye}.
On the other hand, the light sterile neutrinos are generally not natural in the theories that extend the neutrino Standard Model. 
For example, the popular type-I see-saw model~\cite{Minkowski,Yanagida,GellMann,Mohapatra}, which provides an elegant explanation to the small neutrino masses and the matter-antimatter asymmetry of the universe~\cite{Fukugita}, only naturally predicts heavy sterile neutrinos (m $>10^{10}$ eV).
If the light sterile neutrinos indeed exist as LSND indicates, they would suggest new frontiers in both experimental and theoretical physics.

The LSND anomaly is so-far still experimentally unresolved despite the many efforts to do so. 
There are several hints supporting LSND's findings, but are not conclusive.
The MiniBooNE experiment, designed at a similar $L/E$ baseline as LSND using accelerator neutrinos, observed similar event excesses in the $\nu_{\mu}\rightarrow\nu_e$ and $\bar\nu_{\mu}\rightarrow\bar\nu_e$ appearance channels~\cite{MiniBooNE2007,MiniBooNE2013}. 
The GALLEX~\cite{GALLEX2010} and the SAGE~\cite{SAGE2009} solar neutrino experiments, during their calibrations using intense neutrino sources ($^{51}$Cr, $^{37}$Ar), observed a $\sim$24\% deficit in the $\nu_e$ disappearance channel. 
This deficit is often referred to as the ``Gallium anomaly''. 
Recently, re-evaluations of the reactor $\bar\nu_e$ flux resulted in an increase in the predicted $\bar\nu_e$ rate~\cite{Mueller, Huber}. 
Combining the new predictions with the reactor experimental data at baselines between 10--100 m~\cite{ILL,Gosgen,Rovno,Krasnoyarsk,SRP,Bugey4,Bugey3} suggests a $\sim$4--6\% deficit between the measured and predicted reactor $\bar\nu_e$ flux, so-called the ``reactor antineutrino anomaly"~\cite{Mention,Zhang13}. 
These experimental anomalies can be interpreted by additional light sterile neutrinos~\cite{Guinti2011}, but might also be caused by imperfect knowledge of the theoretical predictions or experimental systematics.  
The preferred region ($\Delta{m}^2\sim1$ eV$^2$ and $\sin^22\theta\sim0.01$), however, is in tension with the limits derived from other appearance~\cite{KARMEN2002,NOMAD03,OPERA13,ICARUS13} or disappearance searches~\cite{Stockdale84,Dydak84,MiniBooNE12-nubar,MiniBooNE12-nu,SuperK2000,MINOS11-NC,Bugey3,Conrad12}. In particular, Daya Bay and MINOS+ have recently set stringent limits~\cite{DayaBaySterile,Sousa-Neutrino14} to the allowed regions of light sterile neutrinos.

Due to the strong motivations but rather confusing present experimental status, searching for light sterile neutrinos is a prioritized world-wide program~\cite{sterileWP} with many proposed next-generation neutrino oscillation experiments.
Different technologies will be used, including short-baseline accelerator experiments~\cite{IsoDAR,OscSNS,NESSiE,LAr1-ND,nuSTORM} using various neutrino beams, $^{51}$Cr ($^{44}$Ce-$^{144}$Pr) $\nu_e$ ($\bar\nu_e$) source experiments~\cite{Cribier2011,Dwyer2013,SOX,CeLAND} near large LS detectors, as well as very short baseline ($\sim10$ m) reactor (VSBR) $\bar\nu_e$ experiments. 
In order to unambiguously resolve the LSND anomaly, the oscillation cycles in the $L/E$ space need to be observed, similar as in KamLAND (Fig.~\ref{fig:kamland}) and Daya Bay (Fig.~\ref{fig:dayabay}). VSBR experiments provide unique opportunities given the many advantages provided by reactors.

Multiple VSBR experiments have been proposed globally in the U.S (PROSPECT~\cite{PROSPECT}), Europe (NUCIFER~\cite{NUCIFER-2010, NUCIFER-2014}, STEREO~\cite{NUCIFER-2014}, DANSS~\cite{DANSS}, NEUTRINO-4~\cite{NEUTRINO4-2012,NEUTRINO4-2014}, POSEIDON~\cite{POSEIDON}, SOLID~\cite{SoLid}) and Korea (HANARO~\cite{HANARO}). 
Table~\ref{tab:sterile} summarizes some of the key parameters of the proposed VSBR experiments. 
The oscillation length for the $\sim$1 eV mass-scale sterile neutrinos is about 10 meters for reactor $\bar\nu_e$'s, thus all proposed experiments deploy their detectors at distances of 4-20 m from the reactor cores. 
The reactor cores should ideally be compact in size to minimize the oscillations inside the cores, so most experiments utilize compact research reactors with thermal power of tens of mega-watts. Those research reactors are typically highly enriched in $^{235}$U, in contrast to the commercial reactors in the nuclear power plants. 

\begin{table}[tb]
  \begin{tabular}{lccccc}
  \hline
  & Power & Baseline & Mass & Detector & Seg. \\
  & (MW$_{th}$) & (m) & (ton) &    & \\
  \hline
  PROSPECT  & 85  & 4-20 & 1 \& 10  & $^6$Li-LS & Y \\
  NUCIFER   & 70 & $\sim$7  & 0.7 & Gd-LS & N \\
  STEREO & 57  & $\sim$10 & 1.75  & Gd-LS & N \\
  DANSS & 3000  & 9.7-12.2  & 0.9  & Gd-LS & Y \\
  NEUTRINO-4 & 100  & 6-12  & 1.5  & Gd-LS & N \\
  POSEIDON & 100  & 5-8  & $1.3$ & Gd-LS & N \\
  SOLID & 45-80 & 6.8  & 2.9  & $^6$LiF-ZnS & Y \\
  HANARO & 30  & 6  & $\sim$1  & Gd($^6$Li)-LS & Y \\
  \hline
  \end{tabular}
  \caption{Summary of key parameters of the proposed very short baseline reactor experiments, including reactor thermal power (in mega-watts), distance to reactors, detector target mass, detector technology and whether or not using highly segmented detectors.}
\label{tab:sterile}
\end{table}

Background control is a challenging task in the VSBR experiments. 
The detectors are typically at shallow depth ($\sim$10 m.w.e.)\ constrained by the locations of the reactor cores. 
The cosmic-ray related background is therefore high. 
One advantage of using research reactors is that they can be turned on or off on demands, which helps to measure the reactor-unrelated background. 
The reactor-related backgrounds, such as fast neutrons and high energy gamma rays, are however more difficult to measure as they come in situ with the $\bar\nu_e$ signals. Sufficient active veto and passive shielding are necessary. However, given the tight space near the reactor cores,  they have to be carefully designed.

As shown in Table~\ref{tab:sterile}, The detectors are typically $^{6}$Li-loaded or Gd-loaded liquid (or solid) scintillators. 
The Gd-LS technology is mature and good pulse shape discrimination (PSD) has been demonstrated against the neutron background. One advantage of the $^{6}$Li-loaded scintillator is that the delayed neutron capture process $^{6}$Li$(n,\alpha)t$ produces an $\alpha$ particle instead of a $\gamma$-ray. 
This provides a good localization of the delayed signal and additional PSD against the $\gamma$ background. 
Some detectors are highly segmented into small cells in order to get good position resolution and the ability of background rejection by using the multi-cell event topologies. 
There are however more inactive layers in the segmented detectors so the edge effects have to be accurately simulated.
% in particular for Gd-LS detectors where the delayed neutron capture produces multiple $\gamma$-rays. 
For segmented detectors, it is also more challenging to perform calibrations and control the relative variations between cells. 
For all detectors, sufficient light yield is required to precisely measure the reactor $\bar\nu_e$ spectrum and the possible distortions from neutrino oscillations.

% Thus the PROSPECT experiment is developing $^6$Li-loaded scintillator (Li-LS), while the SOLID experiment will use a composite solid scintillator consists of $^6$LiF-ZnS as neutron layer and plastic scintillator.

Despite the challenges, very short baseline reactor experiments provide an excellent opportunity to observe the distinctive feature of the light sterile neutrino oscillations, due to their extended range of energy (1--8 MeV) and baselines (4--20 m). 
The world-wide next-generation experiments, as shown in Table~\ref{tab:sterile}, are being actively considered and pursued. 
Many of them will begin taking data~\cite{Lhuillier-Neutrino14} in 2015-16. With a few years' running, they expect to cover the parameter region suggested by the experimental anomalies with a sensitivity better than $5\sigma$. 
They may tell us whether the fascinating possibility of light sterile neutrinos is true or not in the very near future.


%!TEX root = reactor_nc_main.tex
%%%%%%%%% Section: Sterile Neutrinos: Very Short Baseline %%%%%%%%%
\section{Searching for Sterile Neutrinos: Very Short Baseline Reactor Experiments}

Neutrino mixing is usually incorporated in a framework in which three-flavor neutrino states, $\nu_e$, $\nu_\mu$ and $\nu_\tau$, are superpositions of the three mass eigenstates $\nu_i$. However, this framework might be incomplete as indicated by several recent experimental results, including the anomalous event excesses in $\bar\nu_e$ appearance experiments (LSND~\cite{LSND2001} and MiniBooNE~\cite{MiniBooNE2013}), the $\nu_e$ rate deficits found in the GALLEX and the SAGE solar neutrino detectors with intense artificial radioactive sources, and the preference for more than three effective number of neutrino species from cosmology surveys (e.g, WAMP~\cite{WMAP2011}, PLANCK~\cite{PLANCK2013}). Those experiments suggest one or more ``sterile" neutrinos might exist and weakly couple to the active neutrinos, although the statistical significance is limited.

The hint of ``sterile" neutrinos also exists in the reactor neutrino experiments. Recently, re-evaluations of the reactor $\bar\nu_e$ flux have showed an increase in the predicted $\bar\nu_e$ rate~\cite{Mueller2011, Huber2011}. Combining with the reactor experimental data at baselines between 10-100m, those calculations suggest a $\sim(5.7\pm2.3)$\% deviation between the measured and predicted reactor $\bar\nu_e$ flux~\cite{Mention2011}, so-called ``reactor anomaly". This reactor anomaly could be a hint of existing additional sterile neutrino states with mass splitting of $\sim1 eV^2$~\cite{Guinti2011}.

However, the reactor anomaly might be imperfect predictions of reactor $\bar\nu_e$ flux. In a later analysis~\cite{Zhang13}, by including the absolute reactor $\bar\nu_e$ flux results from PALO VERDE, CHOOZ and Double CHOOZ at km-scale baselines and using the measured $\theta_{13}$ value from Daya Bay, the new world average ratio of the measured flux to the prediction is only $\sim1.4\sigma$ lower than the unity, thus the significance of the reactor anomaly is weakened. Most recently, the uncertainties in the analysis of the reactor anomaly have been revisited, and it has been found that the components of the aggregate fission spectra containing $\sim30\%$ forbidden decays introduce $\sim$4\% uncertainty in the predicted shape of the reactor $\bar\nu_e$ flux~\cite{Hayes}. This suggests that the ultimate solution to reactor anomaly requires a more precise direct measurement of the antineutrino flux.

Very short baseline reactor experiments have been strongly motivated by the reactor anomaly and the sterile neutrino hypothesis~\cite{sterileWP}, and been proposed worldwide in U.S (PROSPECT~\cite{PROSPECT}), Europe (NUCIFER~\cite{NUCIFER-2010, NUCIFER-2014}, STEREO~\cite{NUCIFER-2014}, DANSS~\cite{DANSS}, NEUTRINO-4~\cite{NEUTRINO4-2012,NEUTRINO4-2014}, POSEIDON~\cite{POSEIDON}, SOLID~\cite{SoLid}) and Korea (HANARO~\cite{HANARO}). Each proposed experiment is located at a research reactors with compact, high $^{235}$U enrichment (HUE) core. The detector in each experimental proposal is deployed at a distance of 5-20 m from the reactor core, with a shallow depth of $\sim$10 m.w.e level.

Good resolution of position and energy in the $\bar\nu_e$ detector is important for the precision oscillation measurement at a meter-scale short baseline. Challenges exists in mitigating the cosmogenic fast neutrons and the reactor-related high energy gammas backgrounds, because of the shallow depth and the short-baseline. Given the tight experimental space near reactor core, sufficient active muon veto and passive shielding structure have to be carefully designed. Highly segmented detectors along the axis pointing toward the core are preferred~\cite{PROSPECT,DANSS,SoLid,HANARO}. The advantages include good position resolution and the ability of background rejection by tagging multi-site event topologies. However, it's challenging to perform calibration and control relative variation of each segmented detector. For the detectors without segmentation~\cite{NUCIFER-2010,NEUTRINO4-2014,POSEIDON} or poor segmentation~\cite{NUCIFER-2014}, the capability of pulse-shape discriminating (PSD) in the gadolinium-loaded LS is important for background rejection. In addition, neutron capture on $^6$Li results in an alpha and a tritium, which can give very discriminant neutron signal. Thus the PROSPECT experiment is developing $^6$Li-loaded scintillator (Li-LS), while the SOLID experiment will use a composite solid scintillator consists of $^6$LiF-ZnS as neutron layer and plastic scintillator.

Very short-baseline reactor experiments have unique signature of new oscillation pattern in a relative large range of energy (1-8 MeV) and baseline (5-20m), which allows good sensitivity to test reactor anomaly and sterile neutrinos with $\delta m^2\sim1$eV$^2$. There are worldwide interests and several experiments are being proposed or under construction. In 2015-16, a few of them will start data taking~\cite{Lhuillier-Neutrino14} and expect to test the reactor anomaly with a sensitivity better than 5$\sigma$ after 1-3 years running.



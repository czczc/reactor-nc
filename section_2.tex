%!TEX root = reactor_nc_main.tex
%%%%%%%%% Section: Flux and Spectrum %%%%%%%%%
\section{Reactor neutrino flux and spectrum} 
\label{sec:flux}

Nuclear reactors derive their power from fission. Both fission fragments are neutron rich and undergo a cascade of $\beta$ decays. Each act of fission
is accompanied by approximately 6 decays, producing an electron and electron antineutrino each. The decay energy, typical for the nuclear
$\beta$ decay, is a few MeV, rarely exceeding $\sim$8 MeV. Since typical power reactors have thermal power of about 3 GW$_{th}$, and produce
$\sim$ 200 MeV of energy in each act of fission, the typical yield of $\bar{\nu}_e$ at equilibrium is $\sim 6 \times 10^{20} \bar{\nu}_e$ core$^{-1}$ s$^{-1}$.   
Reactors are therefore powerful sources of the low energy $\bar{\nu}_e$.

Since the $\bar{\nu}_e$ energy is so low, only charge current reactions producing electrons are possible. Hence, to study neutrino oscillations with
the nuclear reactors, one must use the disappearance type of tests, i.e. measure the flux as a function of the distance $L$ and energy $E_{\nu}$ and
look for the deviation from the simple geometrical scaling. Traditionally, such measurements were compared with the expected  $\bar{\nu}_e$ spectrum
of the reactor. Good knowledge of that spectrum, its normalization and the associated uncertainties is essential in that case.  More recent experiments
\cite{Dayabay,Reno} use two essentially similar detectors, one nearer the reactor and another farther away, in order to reduce the dependence on the
knowledge of the reactor spectrum.

In Box 2 the two principal methods of determining the $\bar{\nu}_e$ spectra associated with fission are briefly described. The summation method was used
initially in \cite{Davis,Vogel81,Klapdor-Pu,Klapdor-U,Kopeikin} and in the updated more recent version in \cite{Mueller, Huber}. The conversion method is based on a series
of measurements of the electron spectra associated with fission \cite{vonFeilitzsch,Schreckenbach,Hahn,Haag}. Naturally, the thermal power of the reactor
and its time changing fuel composition must be known as well as the energy associated with fission of the isotopes $^{235}$U, $^{239}$Pu, $^{241}$Pu
and $^{238}$U. In addition, small corrections to the spectrum shape of individual $\beta$ decay branches due to the radiative correction, weak magnetism,
nuclear size,
etc. must be correctly included. Of a particular importance, but difficult to do accurately, is to take into account the spectrum shape of the numerous
first forbidden $\beta$ decays \cite{Hayes}.

In essentially all reactor neutrino oscillation studies the $\bar{\nu}_e$ are detected using the inverse neutron $\beta$ decay reaction
  \begin{equation}
  \bar{\nu}_e + p \rightarrow e^+ + n~, \\ ~~ \sigma = 9.53 \frac{E_e p_e}{ 1 ~ {\rm MeV^2}} (1 + corrections) \times 10^{-44} {\rm cm^2} ~,
  \label{eq:detection}
  \end{equation}
  whose cross section is accurately known \cite{VB99, Strumia} and depends primarily on the known neutron decay half-life (even though the the recoil, radiative
  corrections etc. must be taken into account as well). Since the neutron is so much heavier than the available energy, its kinetic energy is quite
  small (tens of keV) and thus the principal observables are the number and energy of the positrons. However, the correlated observation of the
  positrons and the delayed neutron captures is a powerful tool for the background suppression. Note that the reaction (\ref{eq:detection}) has
  a threshold of 1.8 MeV, only $\bar{\nu}_e$ with energy larger than that can produce positrons. 
  
  In Fig. \ref{fig:spectra} we illustrate the energy dependence of the reactor $\bar{\nu}_e$ flux, the detection reaction cross section and their product,
  i.e. the observable positron spectrum.
    

\begin{figure}[htb]
\begin{centering}
\includegraphics[width=\columnwidth]{figs/spec.pdf}
\par\end{centering}
\caption{\label{fig:spectra} This is just a placeholder. Reactor $\bar{\nu}_e$ flux, inverse beta-decay cross section,
and the interaction spectrum. This illustration is for  12 ton detector  at 800 m from 12 GW$_{th}$ power reactor. }
\end{figure}

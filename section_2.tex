%!TEX root = reactor_nc_main.tex
%%%%%%%%% Section: Flux and Spectrum %%%%%%%%%
\section{Reactor Neutrino Flux and Spectrum} 
\label{sec:flux}

Nuclear reactors derive their power from fission. Both fission fragments are neutron rich and undergo a cascade of $\beta$ decays. Each act of fission
is accompanied by approximately 6 $\beta$ decays, producing an electron and electron antineutrino each. The decay energy, typical for the nuclear
$\beta$ decay, is a few MeV, rarely exceeding $\sim$8 MeV. Since a typical power reactor core has thermal power of about 3 GW$_{th}$, and produces
$\sim$ 200 MeV of energy in each act of fission, the typical yield of $\bar{\nu}_e$ at equilibrium is $\sim 6 \times 10^{20} \bar{\nu}_e$ core$^{-1}$ s$^{-1}$.   
Reactors are therefore powerful sources of the low energy $\bar{\nu}_e$.

The $\bar{\nu}_e$ energy is low, thus  only charged current reactions producing electrons are possible. Hence, to study neutrino oscillations with
the nuclear reactors, one must use the disappearance type of tests, i.e. measure the flux as a function of the distance $L$ and energy $E_{\nu}$ and
look for the deviation from the simple geometrical scaling. Traditionally, such measurements were compared with the expected  $\bar{\nu}_e$ spectrum
of the reactor. Good knowledge of that spectrum, its normalization and the associated uncertainties is essential in that case.  More recent experiments
\cite{Dayabay,Reno} use two essentially similar detectors, one nearer the reactor and another farther away, in order to reduce the dependence on the
knowledge of the reactor spectrum.

There are two principal, in many ways complementary, ways to evaluate antineutrino spectra associated with fission. 
 The summation method uses known cumulative fission yields $Y_n (Z,A,t)$, and combines them
 with the experimentally known (or theoretically deduced) branching ratios $b_{n,i}(E^i_0)$ of all $\beta$-decay branches with the endpoints $E^i_0$ and the
 normalized shape function of each of these many thousands of $\beta$ decays, $P_{\bar{\nu}} (E_{\bar{\nu}},E^i_0,Z)$,
 \begin{equation}
 \frac{dN}{dE_{\bar{\nu}}} = \Sigma_n Y_n (Z,A,t) \Sigma_i b_{n,i}(E^i_0) P_{\bar{\nu}} (E_{\bar{\nu}},E^i_0,Z) ~.
 \end{equation}
 There are several difficulties with this method. The branching ratios and endpoint energies are sometimes poorly known (or not at all), in particular for
 the short-lived fragments with large $Q$ values and many branches. The individual spectrum shape functions $P_{\bar{\nu}} (E_{\bar{\nu}},E^i_0,Z)$
 require description of the Coulomb distortions including the nuclear finite size effects, weak magnetism, and radiative corrections. In addition, not all
 decays are of the allowed type, there are numerous (about 25\%) first forbidden decays involving parity change, where the individual spectrum 
 shapes are much more difficult to evaluate. 
 
 The other method uses experimentally determined spectrum of electrons associated with fission of the principal reactor fuels. That spectrum has been 
 measured at ILL Grenoble for the thermal neutron fission of $^{235}$U, $^{239}$Pu and $^{242}$Pu and recently also for the fast neutron fission of $^{238}$U
 in Munich. These electron spectra are then transformed into the $\bar{\nu}_e$ spectra using the obvious fact that these two leptons share the total energy
 of each $\beta$ decay branch. The transformation is based on fitting first the electron spectra to a set of 30 or more virtual branches, with the equidistant
 endpoint spacing, determining from the fit their branching ratios. In each of these virtual branches the conversion from the electron to the $\bar{\nu}_e$ is 
 trivial. When putting them together, one has to take into account the fact that different nuclear charges $Z$ contribute with different weights to different electron
 and $\bar{\nu}_e$ energies. The conversion procedure was first used in~\cite{vonFeilitzsch,Schreckenbach,Hahn}, more details can be found in
~\cite{Vogel07} and~\cite{Huber}.
 While the procedure would introduce only minimum uncertainty if all decays would be of the allowed shape, the presence
 of the first forbidden decays again introduces uncertainty whose magnitude is difficult to determine accurately.


The summation method was used
initially in~\cite{Davis,Vogel81,Klapdor-Pu,Klapdor-U,Kopeikin} and in the updated more recent version in~\cite{Mueller, Huber}. The conversion method is based on a series
of measurements of the electron spectra associated with fission~\cite{vonFeilitzsch,Schreckenbach,Hahn,Haag}. Naturally, the thermal power of the reactor
and its time changing fuel composition must be known as well as the energy associated with fission of the isotopes $^{235}$U, $^{239}$Pu, $^{241}$Pu
and $^{238}$U. In addition, small corrections to the spectrum shape of individual $\beta$ decay branches due to the radiative correction, weak magnetism,
nuclear size,
etc. must be correctly included. Of a particular importance, but difficult to do accurately, is to take into account the spectrum shape of the numerous
first forbidden $\beta$ decays~\cite{Hayes}. The overall uncertainty in the flux was estimated in~\cite{Mueller, Huber} to be $\sim$ 2\%, but when the
first forbidded decays are included it is estimated in the Ref.~\cite{Hayes} that the uncertainty increases to $\sim$ 5\%.

In essentially all reactor neutrino oscillation studies the $\bar{\nu}_e$ are detected using the inverse neutron $\beta$ decay reaction
  \begin{equation}
  \bar{\nu}_e + p \rightarrow e^+ + n~, \\ ~ \sigma = 9.53 \frac{E_e p_e}{ 1  {\rm MeV^2}} (1 + corr.) \times 10^{-44} {\rm cm^2} ~,
  \label{eq:detection}
  \end{equation}
  whose cross section is accurately known~\cite{VB99, Strumia} and depends primarily on the known neutron decay half-life (even though the recoil, radiative
  corrections etc. must be taken into account as well). Since the neutron is so much heavier than the available energy, its kinetic energy is quite
  small (tens of keV) and thus the principal observables are the number and energy of the positrons. Most importantly, the correlated observation of the
  positrons and the delayed neutron captures is a powerful tool for the background suppression. Note that the reaction (\ref{eq:detection}) has
  a threshold of 1.8 MeV, only $\bar{\nu}_e$ with energy larger than that can produce positrons. 
  
  In Fig.~\ref{fig:spectra} we illustrate the energy dependence of the reactor $\bar{\nu}_e$ flux, the detection reaction cross section and their product,
  i.e. the measured antineutrino spectrum. The contributions of the individual isotopes to the $\bar{\nu}_e$ flux, weighted by their typical contribution
  to the reactor power are also indicated. On top, the steps involved in the $\bar{\nu}_e$ capture on proton reaction are schematically indicated. 
    

\begin{figure}[htb]
\begin{centering}
\includegraphics[width=\columnwidth]{figs/spec.pdf}
\par\end{centering}
\caption{\label{fig:spectra} Reactor $\bar{\nu}_e$ flux, inverse beta-decay cross section,
and the interaction spectrum. Contributions of the individual isotopes to the flux, weighted by their contribution to the total are also shown.
The steps involved in the reaction (\ref{eq:detection}) are schematically indicated on top of the figure.   }
\end{figure}

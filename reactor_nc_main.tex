\documentclass[aps,twocolumn,preprintnumbers,amsmath,superscriptaddress,amssymb,floats,nofootinbib]{revtex4-1}
\setlength{\topmargin}{-1.cm}

\usepackage{graphicx,color}
%\usepackage{dcolumn}% Align table columns on decimal point
%\usepackage{bm}% bold math
\usepackage[colorlinks=true,citecolor=blue,linkcolor=blue]{hyperref}% PDF links
%\bibliographystyle{unsrt}
\interfootnotelinepenalty=10000

\begin{document}

\title{Reactor Neutrino Oscillation Studies: Past, Present and Future}
\date{\today}
\author{P. Vogel}\email[]{pvogel@caltech.edu}
\affiliation{Kellogg Radiation Laboratory, California Institute of Technology, Pasadena, California}
\author{L. J. Wen}\email[]{wenlj@ihep.ac.cn}
\affiliation{Institute of High Energy Physics, Beijing}
\author{C. Zhang}\email[]{chao@bnl.gov}
\affiliation{Brookhaven National Laboratory, Upton, New York}


%%%%% Preface %%%%%%
\begin{abstract}
Nuclear reactors are one of the most intense, pure, controllable, cost-effective and well-understood sources of neutrinos. Reactors played a major role in the study of the neutrino oscillations, a phenomenon that indicates that neutrinos are not massless and that the neutrino flavors are quantum mechanical mixtures. Over the past several decades reactors were used in the discovery of neutrinos, were crucial in solving the solar neutrino puzzle, and allowed the determination of the smallest mixing angle $\theta_{13}$. In the near future, reactors will help to determine the neutrino mass hierarchy and to solve the puzzling issue of sterile neutrinos.
\end{abstract}

\maketitle
\thispagestyle{plain}

%!TEX root = reactor_nc_main.tex
%%%%%%%%% Section: Intro %%%%%%%%%
\section{Introduction: Neutrino oscillations and nuclear reactors} 
\label{sec:intro}

The Standard Model of electroweak interactions,
developed in late 1960s, incorporates neutrinos as left-handed partners of the charged leptons. Since the weak interactions are the
only way neutrinos interact with anything, the unneeded right-handed components of the neutrino field are absent 
by definition and neutrinos are assumed to be massless with the individual lepton numbers strictly conserved.
This assignment was supported by the lack of observation of decays like 
$\mu^+ \rightarrow e^+ + \gamma$ or $K_L \rightarrow e^{\pm} + \mu^{\mp}$,
despite the long tradition of efforts to observe them.

The discovery over the past several decades
of neutrino oscillations proved that these assumptions were incorrect; it
represents one of the very few instances that show that the otherwise
extremely successful Standard Model is incomplete. It means that
neutrinos have a finite mass, albeit very small, and that the lepton flavor is not a conserved quantity.
Box 1 explains the basic physics of neutrino oscillations and their relation with neutrino masses. It
also introduces the parameters used in the oscillation formalism. Determination of all their values,
with ever increasing accuracy,
was and continues to be the main goal of the experiments.
The current experimental values of the mass-squared differences $\Delta m^2_{ij}$ and of the mixing angles $\theta_{ij}$ can be found
in the latest editions of the Review of Particle Physics~\cite{PDG14}.
Historically, the concept of neutrino oscillations was first considered by Pontecorvo~\cite{Pontecorvo57, Pontecorvo58}
and by Maki, Nakagawa and Sakata~\cite{MNS62}, hence the neutrino mixing matrix is usually called the PMNS matrix.

Study of reactor neutrinos played a very significant part in the discovery and detailed study of neutrino oscillations and will continue 
to be essential to its further progress. Here we briefly review
the main points of this saga. Fig.~\ref{fig:intro1} illustrates how the flavor composition of the reactor neutrino flux, for neutrinos of 4 MeV energy
used as an example, is expected to oscillate as a function of the distance. Experimental verification of this behavior, and the quantitative analysis of the
results are the main topics discussed below.

\begin{figure}[htb]
\begin{centering}
\includegraphics[width=\columnwidth]{figs/osci.pdf}
\par\end{centering}
\caption{\label{fig:intro1} Flavor composition of the 4 MeV energy reactor neutrino flux as a function of the distance  $L$. }
\end{figure}

Existence of neutrinos was predicted by Pauli already in 1930~\cite{Pauli30} in his famous letter attempting to explain the continuous electron
energy distribution in the nuclear beta decay. Only in 1953--1959 Reines and Cowan~\cite{Reines53,Cowan56,Reines59} were able to show that neutrinos
were real particles. Their observation used the electron antineutrinos emitted by a nuclear reactor and started a long tradition of fundamental
discoveries using the $\bar{\nu}_e$ emitted by nuclear reactors.

In the early experiments detectors were placed at distances $L \le 100$ m~\cite{ILL,Gosgen,Rovno,Krasnoyarsk,SRP,Bugey4,Bugey3} (for the review see~\cite{Bemporad02}). As expected, no variations
with the distance were observed, but these pioneering experiments were important for the understanding of the reactor spectrum, discussed
in the next section. The KamLAND experiment~\cite{Kamland03,Kamland05,Kamland08} in 2000s, discussed in more detail in Section III, has convincingly shown that the earlier
solar neutrino measurements are indeed  caused by oscillations. It demonstrated that the reactor neutrinos indeed
oscillate, i.e. that the $\bar{\nu}_e$ component changes as predicted with $L/E_{\nu}$. It also allowed the most accurate determination of the 
mass-squared difference $\Delta m^2_{21}$.  

In the next generation of reactor experiments, Daya Bay~\cite{Dayabay,Dayabay14}, RENO~\cite{Reno}  and Double Chooz~\cite{DChooz,DChooz14}, the longstanding puzzle of the
value of the mixing angle $\theta_{13}$ was successfully resolved; it turns out that its value $\theta_{13} \sim 8.4^\circ$
is not as small as many physicists expected. That discovery, described in Section IV, opened opportunities for further experiments that should  
eventually allow us to determine the so-far missing fundamental features of the oscillations, the neutrino mass hierarchy and the charge-parity violating phase
$\delta_{CP}$. The planned reactor experiment JUNO, described in Section V,  promises to be an important step on that path.

Finally, most of the oscillation results are well described by the simple three-neutrino generations hypothesis. However, there are few anomalous indications,
the so-called reactor antineutrino anomaly~\cite{Mention} among them, that cannot be explained this way. If confirmed, they would indicate the existence of additional
fourth or more neutrino families called sterile neutrinos. They lack  weak interactions and are observable only due to their mixing with the familiar active neutrinos.  Short baseline
reactor experiments, discussed in Section VI, might decide whether this fascinating possibility is realistic or not.

%!TEX root = reactor_nc_main.tex
%%%%%%%%% Box 1 %%%%%%%%%
\onecolumngrid
\vspace{10pt}
\fboxsep=5pt

\noindent\fbox{%
    \parbox{0.98\textwidth}{%


{\Large \textcolor{blue}{Box 1: Neutrino oscillations}}

\setlength{\parindent}{12pt}

Let us assume that there are two massive neutrinos $\nu_i, i=1,2$ with different masses $m_i$. When they propagate in vacuum over a distance $L$,
each acquires the phase $\nu_i(L) = \nu_i (0) \exp(-i m^2_i L/(2E)$. (The overall phase is skipped and it is assumed that the neutrinos are highly relativistic,
$p \sim E - m^2/2E$. Additional phases are acquired when neutrinos propagate in the matter, so-called the ``MSW effect''~\cite{Wolfenstein78,MS85}, which will not be discussed here.) Assume further that the flavor neutrinos $\nu_e$ and $\nu_{\alpha}$, i.e. the neutrinos that are the partners of
charged leptons in the weak interactions, are coherent superpositions of the states $\nu_i$, i.e. $\nu_e = \cos \theta \nu_1 + \sin \theta \nu_2$, and
analogous but orthogonal combination represents the other flavor neutrino $\nu_{\alpha} = -\sin \theta \nu_1 + \cos \theta \nu_2$. This 
mixture is characterized by the parameter $\theta$, so-called mixing angle.  

Consider now a beam of neutrinos that at $L=0$ is pure $\nu_e$. Then
\begin{equation}
\nu_e (L) = \cos \theta e^{-i m_1^2 L/2E} \nu_1(0) + \sin \theta e^{-i m_2^2 L/2E} \nu_2 (0) ~.
\end{equation}
In order to observe this beam at $L$ we must use the weak interactions. Hence we must project the $\nu_i$ back to the flavor basis $\nu_e$ and $\nu_{\alpha}$.
Thus
\begin{equation}
\nu_e (L) = [\cos^2 \theta e^{-i m_1^2 L/2E} + \sin^2 \theta e^{-i m_2^2 L/2E}] \nu_e(0) - 
\sin \theta \cos \theta [e^{-i m_1^2 L/2E} - e^{-i m_2^2 L/2E}] \nu_{\alpha}(0) ~.
\end{equation}
The probability that we detect $\nu_e$ at the distance $L$ is just the square of the corresponding amplitude. The probability
of detecting $\nu_{\alpha}$ is the corresponding square of that amplitude. Therefore, after a bit of a simple algebra,
\begin{equation}
P(\nu_e \rightarrow \nu_e) = 1 - \sin^2 2 \theta \sin^2 \frac{\Delta m^2 L}{4 E} ~,~ P(\nu_e \rightarrow \nu_{\alpha}) =  \sin^2 2 \theta \sin^2 \frac{\Delta m^2 L}{4 E} ~,
\end{equation}
where $\Delta m^2 = m^2_2 - m^2_1$ is the difference of the squares of the neutrino masses. 

Thus, we see that, provided $\Delta m^2 \ne 0$ and if $\theta \ne 0$ or $\pi/2$, the composition of the neutrino beam oscillates as a function of $L/E_{\nu}$ 
with the amplitude $\sin^2 2 \theta$ and wavelength
\begin{equation}
L_{osc} = 4 \pi \frac{E}{\Delta m^2} ~~\equiv ~~ L_{osc} ({\rm m}) = \frac{2.48 E {\rm (MeV)}}{ \Delta m^2 {\rm (eV^2)}} ~. 
\end{equation}

Observation of neutrino oscillations, therefore, constitutes a proof that at least some of the neutrinos have a finite mass and that the superposition is
a nontrivial one. Generalization to the realistic case of three neutrino flavors and three
states of definite mass is straightforward. The corresponding mixing is then characterized by three mixing angles $\theta_{12}, \theta_{13}, \theta_{23}$,
 one possible $CP$ violating phase $\delta_{CP}$,
and two mass square differences $\Delta m^2_{21}$ and  $\Delta m^2_{31}$. The third mass square difference 
$\Delta m^2_{32} = m^2_3 - m^2_2 \equiv \Delta m^2_{31} - \Delta m^2_{21}$ is simply related
to those two. 

Neutrino oscillations have no classical analog. They are purely quantum-mechanical phenomenon, consequence of the coherence of the
superposition of states.

    }%
}

\vspace{10pt}

\twocolumngrid


%!TEX root = reactor_nc_main.tex
%%%%%%%%% Section: Flux and Spectrum %%%%%%%%%
\section{Reactor neutrino flux and spectrum} 
\label{sec:flux}

Nuclear reactors derive their power from fission. Both fission fragments are neutron rich and undergo a cascade of $\beta$ decays. Each act of fission
is accompanied by approximately 6 decays, producing an electron and electron antineutrino each. The decay energy, typical for the nuclear
$\beta$ decay, is a few MeV, rarely exceeding $\sim$8 MeV. Since typical power reactors have thermal power of about 3 GW$_{th}$, and produce
$\sim$ 200 MeV of energy in each act of fission, the typical yield of $\bar{\nu}_e$ at equilibrium is $\sim 6 \times 10^{20} \bar{\nu}_e$ core$^{-1}$ s$^{-1}$.   
Reactors are therefore powerful sources of the low energy $\bar{\nu}_e$.

Since the $\bar{\nu}_e$ energy is so low, only charge current reactions producing electrons are possible. Hence, to study neutrino oscillations with
the nuclear reactors, one must use the disappearance type of tests, i.e. measure the flux as a function of the distance $L$ and energy $E_{\nu}$ and
look for the deviation from the simple geometrical scaling. Traditionally, such measurements were compared with the expected  $\bar{\nu}_e$ spectrum
of the reactor. Good knowledge of that spectrum, its normalization and the associated uncertainties is essential in that case.  More recent experiments
\cite{Dayabay,Reno} use two essentially similar detectors, one nearer the reactor and another farther away, in order to reduce the dependence on the
knowledge of the reactor spectrum.

In Box 2 the two principal methods of determining the $\bar{\nu}_e$ spectra associated with fission are briefly described. The summation method was used
initially in \cite{Davis,Vogel81,Klapdor-Pu,Klapdor-U,Kopeikin} and in the updated more recent version in \cite{Mueller, Huber}. The conversion method is based on a series
of measurements of the electron spectra associated with fission \cite{vonFeilitzsch,Schreckenbach,Hahn,Haag}. Naturally, the thermal power of the reactor
and its time changing fuel composition must be known as well as the energy associated with fission of the isotopes $^{235}$U, $^{239}$Pu, $^{241}$Pu
and $^{238}$U. In addition, small corrections to the spectrum shape of individual $\beta$ decay branches due to the radiative correction, weak magnetism,
nuclear size,
etc. must be correctly included. Of a particular importance, but difficult to do accurately, is to take into account the spectrum shape of the numerous
first forbidden $\beta$ decays \cite{Hayes}.

In essentially all reactor neutrino oscillation studies the $\bar{\nu}_e$ are detected using the inverse neutron $\beta$ decay reaction
  \begin{equation}
  \bar{\nu}_e + p \rightarrow e^+ + n~, \\ ~~ \sigma = 9.53 \frac{E_e p_e}{ 1 ~ {\rm MeV^2}} (1 + corrections) \times 10^{-44} {\rm cm^2} ~,
  \label{eq:detection}
  \end{equation}
  whose cross section is accurately known \cite{VB99, Strumia} and depends primarily on the known neutron decay half-life (even though the the recoil, radiative
  corrections etc. must be taken into account as well). Since the neutron is so much heavier than the available energy, its kinetic energy is quite
  small (tens of keV) and thus the principal observables are the number and energy of the positrons. However, the correlated observation of the
  positrons and the delayed neutron captures is a powerful tool for the background suppression. Note that the reaction (\ref{eq:detection}) has
  a threshold of 1.8 MeV, only $\bar{\nu}_e$ with energy larger than that can produce positrons. 
  
  In Fig. \ref{fig:spectra} we illustrate the energy dependence of the reactor $\bar{\nu}_e$ flux, the detection reaction cross section and their product,
  i.e. the observable positron spectrum.
    

\begin{figure}[htb]
\begin{centering}
\includegraphics[width=\columnwidth]{figs/spec.pdf}
\par\end{centering}
\caption{\label{fig:spectra} This is just a placeholder. Reactor $\bar{\nu}_e$ flux, inverse beta-decay cross section,
and the interaction spectrum. This illustration is for  12 ton detector  at 800 m from 12 GW$_{th}$ power reactor. }
\end{figure}

%!TEX root = reactor_nc_main.tex
%%%%%%%%% Section: KamLAND %%%%%%%%%
\section{Exploring Solar Neutrino Oscillations on Earth: KamLAND} 
\label{sec:kamland}

Since the late 1960s, a series of solar neutrino experiments (Homestake \cite{Homestake}, GALLEX \cite{GALLEX}, SAGE \cite{SAGE}, Kamiokande \cite{Kamiokande}, Super-Kamiokande \cite{Super-Kamiokande}) have observed a large deficit of solar $\nu_e$ flux with respect to the Standard Solar Model (SSM) \cite{Bahcall} prediction. It appeared that more than half of the solar neutrinos were missing. This was referred to as the ``Solar Neutrino Problem''. In 2001, the SNO experiment \cite{SNO} successfully measured the total flux of all three neutrino flavors through the neutral current channel $\nu + d \to \nu + p + n$ using heavy water as a target, and yielded consistent results with the SSM. The SNO result is the ``smoking gun'' evidence of neutrino oscillation explanation to the Solar Neutrino Problem: The solar neutrinos, produced as $\nu_e$ from fusion inside the Sun, have transformed into other flavors when they arrive at the Earth.

The solar neutrino experiments left several possible solutions in the oscillation parameter space of $\theta_{12}$ and $\Delta m^2_{21}$. 
A precise measurement of these parameters and demonstration of the oscillatory feature, however, is hindered by the relatively large uncertainties in the $\nu_{e}$ flux predicted by the SSM, the large matter effect inside the Sun, and the extremely long distance the neutrinos travel. 
A reactor neutrino experiment, measuring the same disappearance channel as the solar neutrino experiments assuming $CPT$ invariance, overcomes these difficulties. 
With well understood and controllable $\sim$MeV $\bar\nu_e$ source, a reactor experiment at $\sim$100 km baselines can explore with high precision the so-called ``Large Mixing Angle (LMA)'' parameter region suggested by the solar neutrino experiments. To do that, the KamLAND experiment \cite{Kamland03} was built in early 2000s to explore the solar neutrino oscillations on Earth.

In order to shield against the cosmic rays, the KamLAND detector was placed at the site of the former Kamiokande experiment \cite{Kamiokande} under the summit of Mt.~Ikenoyama in the Japanese Alps. The vertical overburden is 2700 meter-water-equivalent (m.w.e). It is surrounded by 55 Japanese nuclear reactor cores, which then produced about 30\% of the total electricity in Japan. The $\bar\nu_e$ flux weighted baseline is about 180 km, well suited to study the parameters suggested by the solar neutrino experiments. The reactor operation information such as thermal power, fuel burn-up, and fuel exchange and enrichment records are provided by all Japanese reactors, which allows KamLAND to calculate the instantaneous fission rate of each isotope accurately. 

The KamLAND detector consists of 1 kton of highly purified liquid scintillator (LS), enclosed in a 13-m-diameter transparent balloon suspended by ropes in mineral oil (MO). The MO is housed inside a 18-m-diameter stainless steel (SS) sphere, where an array of 1879 20-inch photomultiplier tubes (PMTs) is mounted. The MO shields the inner LS region from external radiation from PMTs and SS. 3.2 kton of purified water is used to provide further shielding against ambient radiation and operates as an active cosmic muon veto detector. With regular central-axis deployments of radioactive sources and dedicated off-axis deployments, KamLAND achieved a position resolution of $12$ cm$/\sqrt{E(\textrm{MeV})}$, energy resolution of $6.5\%/\sqrt{E(\textrm{MeV})}$ and absolute energy-scale uncertainty of 1.4\%.

Even with such powerful reactor $\bar\nu_e$ sources and a large detector, the long baseline suppresses the expected signal at KamLAND to only about one reactor $\bar\nu_e$ event per day. The experiment is only possible thanks to the powerful coincidence signature (the positron followed by the delayed neutron capture $\gamma$) of inverse beta decay. A time difference of less than 1 millisecond and distance less than 2 meters between the prompt and delayed events is required in the analysis. Only the innermost 6-m radius scintillator region is used to reduce the accidental coincidence from external gamma-rays. Information about the event energy, position and time were used to further reduce the accidental background to $\sim$5\% of the candidates.

The other dominant background ($\sim$10\%) at KamLAND is caused by the $^{13}$C$(\alpha,n)^{16}$O reaction where the $\alpha$-decay comes from $^{210}$Po, a daughter of $^{222}$Rn introduced into the scintillator during construction. The neutron scattering off proton or $^{16}$O$^*$ de-excitation produces a prompt signal, followed by a neutron capture delayed signal, which mimics a true $\bar\nu_e$ event. The rest of the backgrounds include: the antineutrinos produced in the decay chains of $^{232}$Th and $^{238}$U in the Earth's interior (geoneutrinos); the cosmogenic beta delayed-neutron emitters $^{9}$Li and $^{8}$He;  the fast neutrons from muons passing through the surrounding rock, as well atmospheric neutrinos. 

\begin{figure*}[htb] \label{fig:kamland}
  \centering
  \includegraphics[width=0.9\textwidth]{figs/kamland.pdf}
  \caption{KamLAND results: (left) Prompt energy spectrum of $\bar\nu_e$ candidate
  events. All histograms corresponding to reactor spectra and expected
  backgrounds incorporate the energy-dependent selection efficiency
  (top panel). The shaded background and geoneutrino histograms are
  cumulative. 
  % Statistical uncertainties are shown for the data; the band on the blue histogram indicates the event rate systematic uncertainty. 
  (right) Ratio of the background-subtracted
  $\bar\nu_e$ spectrum to the expectation for no-oscillation as a
  function of $L_{0}/E$. $L_{0}$ is the effective baseline taken as a
  flux-weighted average ($L_{0}$\,=\,180\,km). The oscillation survival probability using the best estimates of $\theta_{12}$ and $|\Delta{m}^2_{21}|$ is given by the blue curve.}
\end{figure*}

Fig.~\ref{fig:kamland} (left) shows the prompt energy spectrum of $\bar\nu_e$ candidate
events, observed with 2.9 kton$\cdot$year exposure, overlaid with the expected reactor $\bar\nu_{e}$ and background spectra. A total of 1609 events were observed, which is only about 60\% of the expected signal if there is no oscillation. The ratio of the background-subtracted $\bar\nu_e$ candidate events to no-oscillation expectation is plotted in Fig.~\ref{fig:kamland} (right) as a function of L$_0$/E. The spectrum indicates almost two cycles of the periodic feature as expected from neutrino oscillation, disfavoring other explanations of the $\bar\nu_e$ disappearance.

The KamLAND results \cite{Kamland03,Kamland05,Kamland08} are highly consistent with the solar neutrino experiments, and pinned down the solar neutrino oscillation solution to the LMA region. When combined with the results from SNO, it yields the most precise measurements of $\tan^2\theta_{12} = 0.47^{+0.06}_{-0.05}$ and $\Delta m^2_{21} = 7.59^{+0.21}_{-0.21} \times 10^{-5}$ eV$^2$. It marks the beginning of a precision era in the neutrino oscillation experiments.


%!TEX root = reactor_nc_main.tex
%%%%%%%%% Section: Theta_13 Experiments %%%%%%%%%
\section{Searching for the Smallest Oscillation Angle: $\theta_{13}$ Experiments} 
\label{sec:theta13}
In contrast to the  Cabibbo-Kobayashi-Maskawa (CKM) matrix in quark mixing, where all three mixing angles are very small~\cite{PDG14}, the mixing angles in the neutrino mixing matrix appear to be large: $\theta_{23}$, measured by the atmospheric~\cite{Kajita} and long-baseline accelerator~\cite{Feldman} neutrino experiments, is about $45^\circ$, and $\theta_{12}$, measured by the solar neutrino experiments and KamLAND, is about $33^\circ$. It was therefore natural to expect the third mixing angle, $\theta_{13}$, to be of similar magnitude.

The cleanest way to measure $\theta_{13}$ is through kilometer-baseline reactor neutrino oscillation experiments. The reactor $\bar\nu_e$ oscillation at $\sim$km is dominated by the $\theta_{13}$ terms. Unlike accelerator neutrino experiments, the reactor measurements are independent of the CP-violating phase and $\theta_{23}$, and only slightly dependent on the neutrino mass hierarchy and matter effect. A high precision measurement can therefore be achieved.

In 1990s, two first-generation kilometer-baseline reactor experiments, CHOOZ~\cite{Chooz} and PALO VERDE~\cite{Paloverde} were constructed to measure $\theta_{13}$. 
The CHOOZ detector was built at a distance of $\sim$1050 m from the two reactors of the CHOOZ power plant of \'{E}lectricite\'{d}e France in the Ardennes region of France. It took data from April 1997 until July 1998. 
The PALO VERDE detector was built at distances of 750, 890 and 890 m from the three reactors of the Palo Verde Nuclear Generating Station in the Arizona desert of the United States. It took data between October 1998 and July 2000. 
Surprisingly, neither experiment was able to observe the $\bar\nu_e$ deficit caused by $\theta_{13}$ oscillation. 
As a result, only an upper limit of $\sin^22\theta_{13} < 0.15$ at 90\% C.L. was obtained.

The null results from CHOOZ and PALO VERDE, combined with the measured values of $\theta_{23}$ and $\theta_{12}$, motivated many phenomenological speculations of neutrino mixing patterns such as bimaximal and tribimaximal mixing~\cite{Harrison,Altarelli}. 
In most of these theories, $\theta_{13}$ is either zero or very small. 
A direct consequence of a vanishing $\theta_{13}$ is that the CP violation in the leptonic sector, even if large, can never be observed in the neutrino oscillation experiments. 
The importance of knowing the precise value of $\theta_{13}$ provoked a series of world-wide second-generation kilometer-baseline reactor experiments in the 21st century, including Double Chooz~\cite{DChooz} in France, RENO~\cite{Reno} in Korea and Daya Bay~\cite{Dayabay} in China, to push the sensitivity to $\theta_{13}$ to below $10^\circ$. 
Table~\ref{tab:theta13} summarizes some of the key parameters of the five aforementioned experiments.

\begin{table}[!htb]
  \begin{tabular}{lcccc}
  \hline
  & Power & Baseline & Mass & Overburden \\
  & (GW$_{th}$) & (m) & (ton) & (m.w.e)    \\
  \hline
  CHOOZ        & 8.5  & 1050  & 5    & 300\\
  PALO VERDE   & 11.6 & 750--890  & 12  & 32\\
  \hline
  Double Chooz & 8.5  & 400  & 8  & 120\\
               &      & 1050 & 8  & 300\\
  RENO         & 16.8 & 290  & 16   & 190\\
               &      & 1380 & 16   & 540\\
  Daya Bay     & 17.4 & 360  & 2$\times$20   & 250\\
               &      & 500  & 2$\times$20   & 265\\
               &      & 1580 & 4$\times$20   & 860\\
  % 1998\tablenote{predicted} & 200 & 300 & 1500  & 2000\\
  \hline
  \end{tabular}
  \caption{Summary of key parameters of the reactor $\theta_{13}$ experiments, including reactor thermal power (in giga-watts), distance to reactors, detector target mass and overburden of the underground site (in meter-water-equivalent).}
\label{tab:theta13}
\end{table}

A common technology used in both the first and second generation experiments is the gadolinium-loaded liquid scintillator as $\bar\nu_{e}$ detection target. Gd has a high thermal neutron capture cross section. With $\sim$0.1\% gadolinium loading, the neutron capture time is reduced to $\sim$28 microseconds from $\sim$200 microseconds for the un-loaded scintillator (as is used in KamLAND.) Furthermore, Gd deexcitation after the capture releases an 8-MeV gamma-ray cascade, which gives a delayed signal well above natural radioactivity (In contrast, neutron capture on proton releases a single 2.2-MeV $\gamma$.) The accidental coincidence background is therefore drastically reduced.

The most significant improvement of the second-generation experiments over the previous ones is the addition of near detectors at baselines of a few hundred meters. 
As discussed in Section II, the uncertainty in predicting the reactor antineutrino flux is relatively large (2--5\%.) 
This flux uncertainty, however, can be largely canceled from a relative measurement between near and far detectors. 
The Double Chooz experiment expands CHOOZ by adding a near detector at a distance of $\sim$400 m. 
The installation of the near detector, however, was delayed due to civil construction. 
Double Chooz started taking data in May 2011 with only a far detector, and used the Bugey4~\cite{Bugey4} measurement to normalize the reactor flux. 
The RENO experiment was built near the six reactors of the Yonggwang nuclear power plant in Korea. 
The two identical detectors were located at 290 and 1380 m, respectively, from the center of reactor array. 
RENO started taking data in August 2011. 
The Daya Bay experiment was built near the six reactors of the Daya Bay nuclear power plant in southern China. 
Daya Bay had eight identical antineutrino detectors (ADs). 
Two ADs were placed at $\sim$360 m from the two  Daya Bay reactor cores. 
Two ADs were placed at $\sim$500 m from the four Ling Ao reactor cores. 
And four ADs were placed at a far site $\sim$1580 m away from the 6-reactor complex. 
This modular detector design further allows Daya Bay to largely remove the correlated detector systematics. 
Daya Bay started taking data in December 2011.

Compared to the first-generation experiments, the second-generation experiments have much larger statistics by utilizing higher power reactors and larger detectors. Among them, Daya Bay has the largest reactor power (17.4 GW$_{th}$) and target mass (80 tons at the far site,) as shown in Table~\ref{tab:theta13}. 
The underground sites are much deeper to allow better shielding from cosmogenic background, in particular compared to the case of PALO VERDE. 
Better chemical recipes of the gadolinium-loaded liquid scintillator also improves the overall detector performance and long term stability.

The second-generation reactor experiments were a huge success. 
In 2012, all three experiments, Double Chooz, Daya Bay and RENO, reported clear evidences of $\bar\nu_{e}$ disappearance at $\sim$kilometer baselines with only a few month's running~\cite{DChooz,Reno,Dayabay}. 
In particular, Daya Bay excluded $\theta_{13}=0$ by 5.2 standard deviation with 55 days of data~\cite{Dayabay}. 
The precision of the $\theta_{13}$ measurement improves quickly with more data.
Fig.~\ref{fig:dayabay} (left) from Daya Bay, with the data collected through November 2013~\cite{Zhang-Neutrino14}, shows the ratio of the detected to expected no-oscillation $\bar\nu_{e}$ signals at the 8 detectors located in the three experimental halls, as a function of effective baseline.
The signal rate at the far site shows a clear $\sim$6\% deficit with respect to the near sites, and fits nicely to the theoretical oscillation curve (in red) with $\sin^22\theta_{13} = 0.084 \pm 0.005$. 
Although last known, the precision in $\theta_{13}$ measurement (6\%) is now the best among all three mixing angles.

Similar to KamLAND, the ratio of the detected $\bar\nu_{e}$ events to no-oscillation expectation at Daya Bay is plotted in Fig.~\ref{fig:dayabay} (right) as a function of $L/E_{\nu}$. 
The combined data from the three experimental halls show a near-complete cycle of the expected periodic oscillation feature. 
The smaller amplitude and shorter wavelength of the oscillation, compared to the case of KamLAND, indicate the different oscillation component driven by $\theta_{13}$ and $\Delta{m}^2_{31}$. 
The best-fit frequency of the oscillation yields $|\Delta{m}^2_{31}| = 2.47^{+0.11}_{-0.10} \times 10^{-3}$ eV$^2$ (assuming normal mass hierarchy), which is in good agreement and of comparable precision with the results from atmospheric and long-baseline accelerator neutrino experiments. 

The longstanding puzzle of the value of $\theta_{13}$ is now successfully resolved.
The relatively large value of $\theta_{13}$ opens the gateway for future experiments to determine the neutrino mass hierarchy and measure the CP violating phase in the leptonic sector.

\begin{figure*}[htb] \label{fig:dayabay}
  \centering
  \includegraphics[width=0.95\textwidth]{figs/dayabay.pdf}
  \caption{{\bf Daya Bay results}: (left) Ratio of the detected to expected $\bar\nu_{e}$ signals at the 8 antineutrino detectors (ADs) located in three experimental halls as a function of effective baseline. The oscillation survival probability at the best-fit value is given by the red curve.
  (right) Ratio of the background-subtracted $\bar\nu_e$ spectrum to the expectation for no-oscillation in the three experimental halls, re-expressed as a function of $L_{\textrm{eff}}/E_{\nu}$. The effective baseline $L_{\textrm{eff}}$ is determined for each experimental hall (EH) to an effective oscillated flux from a single baseline. The oscillation survival probability using the best estimates of $\theta_{13}$ and $|\Delta{m}^2_{31}|$ is given by the red curve.}
\end{figure*}






%!TEX root = reactor_nc_main.tex
%%%%%%%%% Section: Mass Hierarchy: JUNO %%%%%%%%%
\newcommand{\fixit}[1]{{\color{red}FIXIT: #1}}
\section{Determination of Neutrino Mass Hierarchy: JUNO}

At present, only the absolute value of the neutrino mass differences $\Delta m^2_{32}$ and $\Delta m^2_{31}$ are known, not their sign. The sign of $\Delta m^2_{31}$ has impacts to many important fundamental topics in particle physics, astrophysics and cosmology. Depending on $\Delta m^2_{31}>0$ or $\Delta m^2_{31}<0$, the neutrino mass ordering is usually referred as normal or inverted mass hierarchy (MH), respectively. The reactor $\bar\nu_e$-oscillation frequency is modulated by $\Delta m^2_{31}$ and $\Delta m^2_{32}$. At a medium baseline of $\sim$60 km, multiple $\theta_{13}$ oscillation peaks are clearly visible on top of the $\theta_{12}$ oscillation, as shown in Figure~\ref{fig:juno}. The MH information can be extracted from the oscillation pattern, by using Fourier transform method~\cite{Zhan-PRD08,Zhan-PRD09} or normal $\chi^2$ analysis~\cite{Li-PRD13}. In addition, the effective mass-squared differences measured by medium-baseline reactor experiment and long-baseline muon neutrino disappearance experiment are different combinations of $\Delta m^2_{31}$, $\Delta m^2_{31}$ and other oscillation parameters. This would provide new information regarding the neutrino MH.

\begin{figure}[!htb] \label{fig:juno}
  \centering
  \includegraphics[width=\columnwidth]{figs/juno.pdf}
  \caption{The reactor $\bar\nu_e$ spectra at a baseline of 60 km in $L/E$ space for no oscillation and $P(\bar\nu_e\to\bar\nu_e)$ oscillation in the cases of normal and inverted neutrino mass hierarchy, assuming $\sin^22\theta_{13}=0.084$.}
\end{figure}

Jiangmen Underground Neutrino Observatory (JUNO), known as Daya
Bay II before, is a multi-purpose experiment with the primary goals to resolve the mass hierarchy issue and precision measurements of neutrino oscillation parameters. JUNO also provide great physics opportunities
in other topics, e.g, observe neutrinos from supernova, Earth interior and Sun, search atmospheric and sterile neutrinos and perform other exotic searches. JUNO locates in Kaiping city, Guangdong province, in south of China, about 150 km west of Hong Kong. The JUNO detector is underground at 480 m below sea level, plus a mountain above it, the total vertical overburden is 1800 m.w.e. JUNO observes $\bar\nu_e$ from Yangjiang nuclear power plant (NPP) and Taishan NPP at a baseline of $\sim$53 km, near the maximum $\theta_{12}$ oscillation. The Yangjiang NPP has six reactors cores of 2.9 GW$_{th}$ thermal power and the Taishan NPP has planned four cores of 4.59 GW$_{th}$, both are under construction. Unequal baselines could cancel the oscillation structure, thus baseline difference is controlled within 500 m to prevent significant degradation of the MH discrimination power~\cite{Li-PRD13}.

The JUNO detectors are under design. The preliminary design is that a central detector submerged in a water pool with muon trackers installed on the top of water pool. The water pool is equipped with PMTs and acts as an active muon veto, and it also provides passive shielding from the natural radioactivities from the surrounding rock and air. The top muon trackers can provide complementary track measurement to the cosmic muons that miss the central detector. The central detector of JUNO consists of 20 kton of un-doped liquid scintillator (LS), contained in a spherical acrylic tank ("Acrylic Sphere" option), or a nylon or other polymer film balloon ("Balloon" option). For the "Acrylic Sphere" option, the acrylic tank is supported by a double layer stainless steel strut. The buffer liquid between the acrylic tank and strut is water, which is physically connected with but optically separated from the outside water Cherenkov detector. For the "Balloon" option, the buffer liquid is non-scintillation LAB contained in a stainless steel sphere of a radius $\sim $40 m. The PMTs are installed on the inner surface of the stainless steel vessel. For both options, the PMTs will be protected from implosion. Taking into account the implosion container and mechanical clearance, the photocathode coverage can reach 75\%-78\% for various options.

Energy resolution is key to JUNO. Th mass hierarchy discrimination power decreases as the energy resolution gets worse~\cite{Li-PRD13}.
It has been found that a $3\%/\sqrt{E}$ energy resolution is essential. Such energy resolution is demonstrated to be achievable by the Monte Carlo simulation, assuming a maximum possible photocathode coverage and a couple of technical improvements: 1) use high quantum efficiency (QE) PMTs with the maximum QE to be 35\%; 2) improve the LS attenuation length to 20 m at 430nm, which consists of an absorption length of 60 m and a Rayleigh scattering length of 30 m. A new type of 20 inch, high efficiency PMT is being developed by JUNO. Instead of using the traditional dynode, the new PMT design uses micro-channel plate (MCP) to have nearly 4$\pi$ collection of the photoelectrons. Using the super bialkali technic, the quantum efficiency of the photocathode is expected to reach to 35\% high. To increase the optical transparency of LS, purification of the raw solvent and fluors at the production stage and online purification such as Al$_2$O$_3$ column filtration at the operation stage is effective. A position resolution of \fixit{?cm}/$\sqrt{E (MeV)}$ is achievable by using time-based likelihood reconstruction method.

%Other effects like the uncertainties from non-uniformity correction, PMT dark rates, electronics noise, etc, would also affect the energy resolution and should be carefully considered during the detector design.

Unlike the $\theta_{13}$ reactor experiments, JUNO uses un-loaded scintillator to reduce the risk in the production, long-term operation, as well as the online purification of a massive detector. Increasing the light yield and optical transparency of the liquid scintillator is important for improving the energy resolution. Maximum light yield can be achieved by optimizing the concentration of fluors. The other issue is to reduce the radioactive impurities in LS, particularly for the potential research of solar neutrinos. An online purification system with distillation is essential to remove U/Th and lead-210 from LS.

%In JUNO, the mass hierarchy information is extracted from the oscillation structure in the energy spectrum.
Three main effects can cause non-linear energy response in a LS detector: quenching effect, Cerenkov process and possible electronics non-linear response. If the energy non-linearity correction has large uncertainties, particular residual non-linearity shape can fake the oscillation pattern with a wrong mass hierarchy~\cite{Qian-PRD13}. The energy scale uncertainty need be controlled within 1\%, which is expected to be achievable in JUNO since it has been achieved in a smaller detector like Daya Bay~\cite{Zhang-Neutrino14}. It is found that the multiple $\theta_{13}$ oscillation peaks in the measured $L/E$ spectrum can be utilized to constrain the non-linearity response formula~\cite{Li-PRD13}.

With full operation of the reactors from Yangjiang and Taishan NPP, the expected signal at JUNO is 80 reactor $\bar\nu_e$ events per day. The delayed signal is a single 2.2-MeV $\gamma$ released by neutron capture on protons with a capture time of $\sim$200 microseconds. Though the natural radioactivity from detector materials can contaminate the delayed signal, coincidences in energy, position and time can suppress the accidental background to $\sim$2\% of the $\bar\nu_e$ candidates. The cosmogenic $\beta$-n emitters $^9$Li and $^8$He is the other dominant background at JUNO, due to large detector size and relative high cosmic muon rate. Similar as previous generation reactor neutrino experiments, the rest of the backgrounds at JUNO include fast neutrons, the $^{13}$C$(\alpha, n)^{16}$O background, geo-neutrinos.

The sensitivity of mass hierarchy determination at JUNO can exceed 3$\sigma$ after taking into account the realistic locations of the reactor cores, uncertainties of the energy non-linearity, etc. Assume the effective mass-squared difference measured by accelerator experiments can reach to $\sim$1.5\% or 1\% precision~\cite{Agarwalla}, then it can help to improve sensitivity to be 3.7$\sigma$ or 4.4$\sigma$\cite{Li-PRD13}. Besides, precision measurement of the neutrino oscillation parameters allows testing unitarity of the neutrino mixing matrix, which is of great importance. For $\Delta m^2_{21}$, $\Delta m^2_{32}$ and $\sin^2\theta_{12}$, JUNO expects to measure them with a precision better than 1\%. Considering the precision of $\sin^2\theta_{13}$ can be measured to $\sim4\%$ by the on-going $\theta_{13}$ reactor experiments, the unitarity of the neutrino mixing matrix can be probed to 1\% level.

Measuring mass hierarchy at JUNO with reactors is complementary in physics since it's independent on $\theta_{23}$ and the unknown CP phase. The technical challenges are significant to build a 20kton liquid scintillator detector with high precision and excellent energy resolution. JUNO expects to start data taking in 2020.

%!TEX root = reactor_nc_main.tex
%%%%%%%%% Section: Sterile Neutrinos: Very Short Baseline %%%%%%%%%
\section{Searching for Sterile Neutrinos: Very Short Baseline Reactor Experiments}

The number of light neutrino species that determined from the invisible decay width of the Z boson by combining all the experiments in the 1990s at the electron-position colliders SLC and LEP~\cite{EW-2005} is in agreement with the three observed flavors of neutrino. Whether there exist extra species of neutrinos with no ordinary weak interactions, so-called sterile neutrinos, is one of the fundamental questions in neutrino physics and cosmology.

In the past decade, the picture of neutrino oscillations has been established and the mass squared difference $\delta m^2$ can be measured in different oscillation channels. The solar-neutrino~\cite{SNO} and the reactor-antineutrino~\cite{Kamland03} experiment has observed the neutrino oscillation in $\nu_e$ and $\bar\nu_e$ disappearance channel at $\delta m^2\sim8\times10^{-5}$eV$^2$. The atmospheric-neutrino~\cite{SuperK98} and long-baseline accelerator-neutrino experiments~\cite{K2K-2003,MINOS06} has observed the neutrino oscillation in $\nu_\mu$ disappearance channel at $\delta m^2\sim3\times10^{-3}$eV$^2$. In the early 2000s, the LSND experiment~\cite{LSND2001} reported anomalous event excesses in the $\bar\nu_\mu\rightarrow\bar\nu_e$ appearance channel, and interpreted it as an evidence for $\bar\nu_\mu\rightarrow\bar\nu_e$ oscillations at the $\delta m^2\sim1$eV$^2$ scale. However, such excess was not confirmed by a similar experiment KARMEN~\cite{KARMEN2002}.

These three different $\delta m^2$ scales do not agree with the Standard Model which requires three neutrino species. The phenomenological models introducing one or more additional ``light" sterile neutrinos (due to their small masses $\lesssim$1 eV)~\cite{Serel04} could explain the tension. However, the light sterile neutrinos are less motivated, comparing to the heavy sterile neutrinos that can provide an elegant interpretation of the small neutrino masses~\cite{SeeSaw} and contribute to the mechanism for matter-antimatter asymmetry of the universe~\cite{leptogenesis}. Thus the tension should be addressed experimentally. The MiniBooNE experiment~\cite{MiniBooNE2007} was motivated to check the results from LSND, and its recent results did not refute LSND~\cite{MiniBooNE2013}. Interestingly, most recent MINOS data has shown no evidence for $\nu_\mu$ disappearance into sterile neutrinos at $\delta m^2<$1 eV$^2$~\cite{Sousa-Neutrino14}. Furthermore, future short-baseline accelerator-neutrino experiments~\cite{nuSTORM,IsoDAR} are being proposed to solve the sterile neutrino issue.

%Sterile neutrino~\cite{sterileWP} have been an active field in the past decade, the fundamental issue is that

%Neutrino mixing is usually incorporated in a framework in which three-flavor neutrino states, $\nu_e$, $\nu_\mu$ and $\nu_\tau$, are superpositions of the three mass eigenstates $\nu_i$. However, this framework might be incomplete as indicated by several recent experimental results, including the anomalous event excesses in $\bar\nu_e$ appearance experiments (LSND~\cite{LSND2001} and MiniBooNE~\cite{MiniBooNE2013}),

The $\nu_e$ rate deficits have been found in the GALLEX~\cite{GALLEX,GALLEX2010} and the SAGE~\cite{SAGE,SAGE2009} solar neutrino detectors with intense artificial radioactive sources, often referred as ``Gallium anomaly". This could be explained by the oscillation caused by sterile neutrino~\cite{Guinti2010}, although the significance is about $\sim2\sigma$. To unambiguously clarify this anomaly, new experiments are proposed, using a very intense $^{51}$Cr or $^{44}$Ce-$^{144}$Pr source~\cite{Cribier2011,Dwyer2013,SOX,CeLAND} next to a large LS detector.

%In addition, most recent cosmology surveys (e.g, WAMP~\cite{WMAP2011}, PLANCK~\cite{PLANCK2013,PLANCK-BICEP2}) also showed hits of more than three effective number of neutrino species.

The hint of sterile neutrinos also exists in the reactor neutrino experiments. Recently, re-evaluations of the reactor $\bar\nu_e$ flux have showed an increase in the predicted $\bar\nu_e$ rate~\cite{Mueller2011, Huber2011}. Combining with the reactor experimental data at baselines between 10-100m, those calculations suggest a $\sim(5.7\pm2.3)$\% deviation between the measured and predicted reactor $\bar\nu_e$ flux~\cite{Mention2011}, so-called ``reactor anomaly". This reactor anomaly could be a hint of existing additional sterile neutrino states with mass splitting of $\sim1$ eV$^2$~\cite{Guinti2011}.

However, the reactor anomaly might be caused by imperfect predictions of reactor $\bar\nu_e$ flux. In a later analysis~\cite{Zhang13}, by including the absolute reactor $\bar\nu_e$ flux results from PALO VERDE, CHOOZ and Double CHOOZ at km-scale baselines and using the measured $\theta_{13}$ value from Daya Bay, the new world average ratio of the measured flux to the prediction is only $\sim1.4\sigma$ lower than the unity, thus the significance of the reactor anomaly is weakened. In the past year the uncertainties in the analysis of the reactor anomaly have been revisited, and it has been found that the components of the aggregate fission spectra containing $\sim30\%$ forbidden decays introduce $\sim$4\% uncertainty in the predicted shape of the reactor $\bar\nu_e$ flux~\cite{Hayes}. This suggests that the ultimate solution to reactor anomaly requires a more precise direct measurement of the antineutrino flux. Most recently, the sterile neutrino search at Daya Bay using only the relative spectral distortion~\cite{DayaBaySterile} showed no evidence in the $10^{-3}$ eV$^2<\delta m^2<0.3$ eV$^2$ range.

Very short baseline reactor experiments have been strongly motivated by the reactor anomaly and the sterile neutrino hypothesis~\cite{sterileWP}, and been proposed worldwide in U.S (PROSPECT~\cite{PROSPECT}), Europe (NUCIFER~\cite{NUCIFER-2010, NUCIFER-2014}, STEREO~\cite{NUCIFER-2014}, DANSS~\cite{DANSS}, NEUTRINO-4~\cite{NEUTRINO4-2012,NEUTRINO4-2014}, POSEIDON~\cite{POSEIDON}, SOLID~\cite{SoLid}) and Korea (HANARO~\cite{HANARO}). Each proposed experiment is located at a research reactors with compact, high $^{235}$U enrichment (HUE) core. The detector in each experimental proposal is deployed at a distance of 4-20 m from the reactor core, with a shallow depth of $\sim$10 m.w.e level. Table ~\ref{tab:sterile} summarizes some of the key parameters of the worldwide proposed very short baseline reactor experiments.

\begin{table}[t]
  \begin{tabular}{lcccc}
  \hline
  & Power & Baseline & M$_{target}$ & Overburden \\
  & (MW$_{th}$) & (m) & (ton) & (m.w.e)    \\
  \hline
  PROSPECT  & 85  & 4-20 & 1 \& 10     & $<$10 \\
%  \hline
  NUCIFER   & 70 & $\sim$7  & 0.7 & 13\\
  STEREO & 57  & $\sim$10 & 1.75  & 15\\
  DANSS & 3000  & 9.7-12.2  & 0.9  & 50\\
  NEUTRINO-4 & 100  & 6-12  & 1.5  & $\sim$10 \\
  POSEIDON & 100  & 5-8  & $\sim1.3$ & few\\
  SOLID & 45-80 & 6.8  & 2.9  & 10\\
%  \hline
  HANARO & 30  & 6  & $\sim$1  & $<$10\\
  \hline
  \end{tabular}
  \caption{Summary of key parameters of the proposed very short-baseline reactor experiments, including reactor thermal power (in giga-watts), distance to reactors, detector target mass and overburden of the underground site (in meter-water-equivalent).}
\label{tab:sterile}
\end{table}

Good resolution of position and energy in the $\bar\nu_e$ detector is important for the precision oscillation measurement at a meter-scale short baseline. Challenges exists in mitigating the cosmogenic fast neutrons and the reactor-related high energy gammas backgrounds, because of the shallow depth and the short-baseline. Given the tight experimental space near reactor core, sufficient active muon veto and passive shielding structure have to be carefully designed. Highly segmented detectors along the axis pointing toward the core are preferred~\cite{PROSPECT,DANSS,SoLid,HANARO}. The advantages include good position resolution and the ability of background rejection by tagging multi-site event topologies. However, it's challenging to perform calibration and control relative variation of each segmented detector. For the detectors without segmentation~\cite{NUCIFER-2010,NEUTRINO4-2014,POSEIDON} or poor segmentation~\cite{NUCIFER-2014}, the capability of pulse-shape discriminating (PSD) in the gadolinium-loaded LS is important for background rejection. In addition, neutron capture on $^6$Li results in an alpha and a tritium, which can give very discriminant neutron signal. Thus the PROSPECT experiment is developing $^6$Li-loaded scintillator (Li-LS), while the SOLID experiment will use a composite solid scintillator consists of $^6$LiF-ZnS as neutron layer and plastic scintillator.

Very short-baseline reactor experiments have unique signature of new oscillation pattern in a relative large range of energy (1-8 MeV) and baseline (4-20m), which allows good sensitivity to test reactor anomaly and sterile neutrinos with $\delta m^2\sim1$eV$^2$. There are worldwide interests and several experiments are being proposed or under construction. In 2015-16, a few of them will start data taking~\cite{Lhuillier-Neutrino14} and expect to test the reactor anomaly with a sensitivity better than 5$\sigma$ after 1-3 years running.



%!TEX root = reactor_nc_main.tex
%%%%%%%%% Section: Conclusions %%%%%%%%%
\section{Prospects} 
\label{sec:prospects}

Over the past $\sim$60 years, nuclear reactors have proven to be one of the most powerful tools to study neutrino oscillations, the quantum-mechanical phenomenon that requires extensions to the Standard Model. Experiments at a few kilometers and at a few hundred kilometers from the reactor cores produced the most convincing proofs of neutrino oscillations, by observing in the $L/E$ domain the oscillatory behavior of reactor $\bar\nu_e$'s during their propagations. Several key parameters governing the neutrino mixing, $\theta_{12}$, $\theta_{13}$, $\Delta{m}^2_{21}$ and $|\Delta{m}^2_{31}|$, were also precisely measured by reactor experiments. With the upcoming next-generation reactor neutrino oscillation experiments, we expect to uncover more facts about neutrinos, which may hold the key to our deeper understanding of the fundamental physics and the universe.



\section*{Acknowledgment}
We thank X.~Qian and D.~Jaffe for reading the manuscript.
The work of C.Z.~was supported in part by the Department of Energy under contracts DE-AC02-98CH10886.
The work of P.V.~was supported in part by the National Science Foundation NSF-1205977 and by the Physics Department, California Institute of Technology.
The work of L.J.W.~was supported in part by National Natural Science Foundation of China (Y3118S005C).

\bibliographystyle{naturemag}
\bibliography{references}



\end{document}



%!TEX root = reactor_nc_main.tex
%%%%%%%%% Box 2 %%%%%%%%%
\onecolumngrid
\vspace{10pt}

\noindent\fbox{%
    \parbox{0.98\textwidth}{%

{\Large \textcolor{blue}{Box 2: Reactor $\bar{\nu}_e$ flux and spectrum.}}


 There are two principal ways to evaluate antineutrino spectra associated with fission. 
 The summation method uses known cumulative fission yields $Y_n (Z,A,t)$, and combines them
 with the experimentally known (or theoretically deduced) branching ratios $b_{n,i}(E^i_0)$ of all decay branches with the endpoints $E^i_0$ and the
 normalized shape function of each of these many thousands of $\beta$ decays, $P_{\bar{\nu}} (E_{\bar{\nu}},E^i_0,Z)$,
 \begin{equation}
 \frac{dN}{dE_{\bar{\nu}}} = \Sigma_n Y_n (Z,A,t) \Sigma_i b_{n,i}(E^i_0) P_{\bar{\nu}} (E_{\bar{\nu}},E^i_0,Z) ~.
 \end{equation}
 There are several difficulties with this method. The branching ratios and endpoint energies are sometimes poorly known (or not at all), in particular for
 the short-lived fragments with large $Q$ values and many branches. The individual spectrum shape functions $P_{\bar{\nu}} (E_{\bar{\nu}},E^i_0,Z)$
 require description of the Coulomb distortions including the nuclear finite size effects, weak magnetism, and radiative corrections. In addition, not all
 decays are of the allowed type, there are numerous (about 25\%) first forbidden decays involving parity change, where the individual spectrum 
 shapes are much more difficult to evaluate. 
 
 The other method uses experimentally determined spectrum of electrons associated with fission of the principal reactor fuels. That spectrum has been 
 measured at ILL Grenoble for the thermal neutron fission of $^{235}$U, $^{239}$Pu and $^{242}$Pu and recently also for the fast neutron fission of $^{238}$U
 in Munich. These electron spectra are then transformed into the $\bar{\nu}_e$ spectra using the obvious fact that these two leptons share the total energy
 of each $\beta$ decay branch. While the procedure would introduce only minimum uncertainty if all decays would be of the allowed shape, the presence
 of the first forbidden decays again introduces uncertainty whose magnitude is difficult to determine accurately.

     }%
 }

 \vspace{10pt}

 \twocolumngrid
